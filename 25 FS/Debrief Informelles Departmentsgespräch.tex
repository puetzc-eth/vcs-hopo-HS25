\documentclass{article}
\usepackage{graphicx} % Required for inserting images
%Welcome :)

\documentclass{article}

% Basic document formatting
\usepackage[utf8]{inputenc}     % Input encoding
\usepackage[T1]{fontenc}        % Font encoding
\usepackage{lmodern}            % Modern LaTeX fonts
\usepackage{geometry}           % Set page margins
\geometry{a4paper, total={170mm,257mm}, left=20mm, top=20mm} 
\usepackage{float}              % Handling of floating elements
\usepackage{fancyhdr}           % Fancy headers
\usepackage{lastpage}           % use \pageref{LastPage} to make page x of y footers
\setlength{\parindent}{0pt}     % No \noindent

% Figures
\usepackage{graphicx}           % For including images
\usepackage{caption}            % Using the caption package
\usepackage{wrapfig}            % For including Wrap Figures
\usepackage{subcaption}         % For subfigures within a figure environment
\usepackage{pgfplots}           % Drawing plots
\usepackage{pgf-pie}            % For creating pie charts
\captionsetup[figure]{labelfont=bf}
\captionsetup[table]{labelfont=bf}
\usepackage{asymptote}          % Zum Zeichnen verschiedener Plots 
\usepackage{pdfpages}           % Zum Einfügen ganzer PDFs

% Colorboxes
\usepackage[skins]{tcolorbox}   % Color Boxes

% Tables and long tables
\usepackage{tabularx}           % Advanced table features
\usepackage{longtable}          % For tables that span multiple pages
\usepackage{multirow}           % Allows for multirow cells in tables
\usepackage{booktabs}           % For professional-quality tables

% Math packages
\usepackage{amsmath}            % Enhanced mathematical formatting
\usepackage{amssymb}            % Extended symbol collection
\usepackage{amsfonts}           % Mathematical fonts
\usepackage[version=4]{mhchem}  % Chemische Formeln
\usepackage{mathtools}          % Mathematical tools to supplement amsmath
\numberwithin{equation}{section} % Numbers Equations with chapters
\usepackage{siunitx}            % Makes SI-Units

% Code display
\usepackage{listings}           % For displaying code
\usepackage{xcolor}             % For coloring code
\lstdefinestyle{mystyle}{
    backgroundcolor=\color{backcolour},   
    commentstyle=\color{codegreen},
    keywordstyle=\color{ao},
    numberstyle=\tiny\color{codegray},
    basicstyle=\ttfamily\footnotesize,
    breakatwhitespace=false,         
    breaklines=true,                 
    captionpos=b,                    
    keepspaces=true,                 
    numbers=left,                    
    numbersep=5pt,                  
    showspaces=false,                
    showstringspaces=false,
    showtabs=false,                  
    tabsize=2
}
\lstset{style=mystyle}

% Custom Colours
\definecolor{LightCyan}{rgb}{0.88,1,1}
\definecolor{dkgreen}{rgb}{0,0.6,0}
\definecolor{gray}{rgb}{0.5,0.5,0.5}
\definecolor{mauve}{rgb}{0.58,0,0.82}
\definecolor{codegreen}{rgb}{0,0.6,0}
\definecolor{codegray}{rgb}{0.5,0.5,0.5}
\definecolor{ao}{rgb}{0.0, 0.0, 1.0}
\definecolor{backcolour}{rgb}{0.95,0.95,0.92}

% Referencing
\usepackage[style=numeric, backend=biber, sorting=none]{biblatex} % Imports biblatex package
\addbibresource{MAIN.bib} %Import the bibliography file
\DeclareFieldFormat{labelnumberwidth}{\mkbibbrackets{#1}} %ensure that the label numbers in the bibliography are enclosed in brackets.
\usepackage{xurl}

% Hyperlinks in the document
\usepackage{hyperref}           % For adding hyperlinks


\begin{document}

\begin{center}
  \begin{minipage}{0.24\textwidth}
    \centering
    \includegraphics[width=\textwidth]{Graphics/VCS-Logo.png}
  \end{minipage}
  \hfill
  \begin{minipage}{0.24\textwidth}
    \centering
    \includegraphics[width=0.75\textwidth]{Graphics/NewLogo_PNG.png}
  \end{minipage}
  \hfill
  \begin{minipage}{0.24\textwidth}
    \centering
    \includegraphics[width=0.75\textwidth]{Graphics/VAC_logo_vector-2.jpeg}
  \end{minipage}
  \hfill
  \begin{minipage}{0.24\textwidth}
    \centering
    \includegraphics[width=0.9\textwidth]{Graphics/PSA_logo.jpg}
  \end{minipage}
\end{center}

\vspace{-0.5cm}
\begin{flushright}
\parbox[r]{4cm}{Zurich, \today}
\end{flushright}

\vspace{0.2cm}
\begin{center}
\textbf{\LARGE{DEBRIEF Informelles Departementsgespräch (D-CHAB)}} \\
\end{center}

Anwesende, D-CHAB: Copéret, Regula Merz, Daniel Ivernot-Perez, Jeschke
Awwesende, Fachvereine: Connor und Ellie (VSC), Lorenzo Martinoli und Olivia Bauert (APV) und Jacob Spaeth (PSA); (Lea Marti vom VAC war krank) 
[Handlungsbedarf in \textbf{fettgedruckt}.]
\section{Finanzen (VCS)}

\subsection{Druck Null Exsi - Fachvereinszeitschrift}
\begin{itemize}
    \item wir sollen erstmal ein grobes \textbf{5-Jahres Budget austellen} für alle Beiträge, die das D-CHAB uns beisteuert/ vorraussichtlich beisteuern wird (\textit{Copperet will uns dazu "motivieren"})
    \item wird von allen Fachvereinen verlangt, die Geld vom D-CHAB kriegen, also VCS, APV und VAC
    \item für die letzten 17 Jahre (in Durchschnitten) wurde ne \textbf{Tabelle} gezeigt --> \textbf{HIER EINFÜGEN}
    \item generell Verständnis für gebrauchte Unterstützung bei der Finanzierung, wenn die Studierendenzahl wächst. Aber erstmal Hausaufgaben bzgl Budget machen... damit sie langfristiger Planen können
\end{itemize}

\subsection{Marco Ponts Award (VCS)}
\begin{itemize}
    \item es wird KEINE Finanzierung für einen Geld-Preis vom Department geben (das sei schon von Anfang an klar so gewesen)
    \item Coperet will Gratulationsschreiben machen
    \item lange Diskussion, hat aber leider nix gebracht
    \item itme APV hat auch versichert, dass es Ihnen nichts ausmacht, wenn sie kein Geld kriegen, weil es bei ihnen nicht viele TAs gibt. Das wurde aber kaum aufgenommen, da ja in ein paar Jahren sich die Meinung des APVs ja auch ändern könnte und dann wär es ja ungerecht/ "unsymmetrisch". Ig sie wollen einfach kein Geld reinstecken.
\end{itemize}

\section{PAKETH}
Die immer näher rückende Umsetzung von PAKETH sorgt bei Studierenden für Fragen über die konkrete Umsetzung am D-CHAB.

\subsection{Buddy-System für Erstis (VCS \& APV)}
\begin{itemize}
    \item Positiv aufgenommen (besonders, dass es von APV \& VCS zusammen kommt, meinte Regula)
    \item noch keine konkreten Vereinbarungen, wann was passiert
\end{itemize}

\subsection{Studiengangsvernetzungstreffen (VCS)}
\begin{itemize}
    \item Hemmungen, sich um die Organisation zu kümmern oder Geld zu geben (vermutlich auch nicht für kleinen Apero)
    \item Jeschke als Studienleiter Chemie schien durchaus motiviert
    \item coperet hat vorgeschlagen, dass es (wie bei den phd colloquien) von den Studierenden aus gehen sollte und sie die Profs/ Studiendirektor*innen einladen (mit verschiedenen Terminvorschlägen)
    \item (bei APV nicht wirklich bedarf da sich eh alle kennen)
    \item Ellies Idee: von Semestersprechenden koordinieren lassen? --> \textbf{Rücksprache mit Semestersprechenden}
    \item location vermutlich auch am einfachsten zwischen fingern, draussen
    \item Jeschke meinte sich zu erinnern, dass es sowas (im HXE) 2019 für die Chems gegeben hatte
\end{itemize}

\subsection{Notenbekanntgabe unter PAKETH (VCS)}
\begin{itemize}
    \item tjaa wird schwierig
    \item hängt noch zu grossem Teil von der digitalen Infrastruktur ab, v.a. wie schnell sie sich die Notenverteilungen anschauen können und dann evtl. was ändern müssen
    \item wollen es auch, besonders im HS immer ASAP hinkriegen, sehen die challenge
    \item haben sich bedankt für unsere Ansicht/ Bemerkung, dass die Noten am Semesteranfang schon wichtig sind
\end{itemize}

\subsection{Workload Analyse (VCS)}
\begin{itemize}
    \item Jeschke würde sich SEHR über die daten freuen --> @Mathilde \textbf{Jescke die Workload-Stunden vom Semesterfeedback schicken}
    \item es existiert anscheinend bei Regula ein Dokument mit semi-brauchbaren Daten mit WOrkload-Daten von den Unterrichts- und Prüfungsevaluationen (welche abwechslend erhoben werden, also keine zusammenhängenden daten für ein Fach ergeben LoL, zumindest schien es so) --> \textbf{haben wir auch Interesse, das Dokument mal zu sehen, wenn Regula es rausrückt?}
\end{itemize}

\subsection{Video-Aufzeichnungen (VCS)}
\begin{itemize}
    \item Punkt übersprungen, liegt halt in der Hand der Dozent:innen
\end{itemize}\textbf{}

\subsection{Repetitionswoche im HS (VCS)}
\begin{itemize}
    \item es wird als unrealistisch angesehen, dass die D-CHAB-Profs schon gemäss Weisung des Rektors ab HS26 mit PAKETH-angepasstem Curriculum agieren (sie haben ein bisschen drüber gelacht)
    \item werden es aber als Anreiz für interessierte und neue Profs in die AB mitnehmen
    \item APV hat sowas schon in mehreren Vorlesungen, von den Profs selsbt initiiert (und Lehrplan verschlankt dafür) - bei manchen in Woche 7 oder 8, bei anderen in letzter SW
\end{itemize}

\section{Transcript of Teaching (VAC)}
\begin{itemize}
    \item generell positiv aufgenommen, aber erstmal unbestimmt in die Zukunft gesetzt, da es mit den ganzen Digital Campus Umstellungen dann kommen sollte (wissen anscheinend selbst noch nicht genau, wann)
    \item coperet hat auch gesagt, dass er gut findet, wenn man auf dem abschlusszeugnis (oder whatever) sieht, dass die D-CHAB PHDs actually teaching gemacht haben und was genau, nicht wie an anderen Departementen (shots fired at D-INFK, glaub ich)
    \item sie haben auch nichts dagegen, dass es entsprechend auch bei Bachelor und Masterabschlüssen vermerkt wird
\end{itemize}

\section{Weiteres}
\subsection{Transparenz - Kosten der Praktikas (APV \& VCS)}
\begin{itemize}
    \item gernerell Unwilligkeit, was zu tun bzw. an dem Dokument auf der ETH website zu ändern
    \item bei der letzen Umfrage (2013) an die Departemente waren die Laborkosten laut Regula immernoch in der angegebenen Range, deswegen wurde nix extra fürs D-CHAB aufgelistet
    \item aktueller Stand der Kostenübersicht (Januar 2024) ist aber OHNE irgendeine Kategorie für Laborkosten --> es wird also nicht ersichtlich, dass man das selber zahlen muss (was ja unser Punkt war)
    \item kritisierten, dass wir keine Zahlen haben, ob es denn wirklich ausserhalb der Range sei - aber an sich könnte zB Regula oder auch Daniel ja viel leichter an die Daten kommen, wenn sie Kontakt zu den Lab-Koordinator:innen aufnehmen würden (haben hierzu keinerlei Initiative oder Lust gezeigt, wird wohl an uns hängen bleiben)
    \item immerhin: \textbf{siehe Mail von Regula, Fr 16.04., "Re: Studien- und Lebenshaltungskosten - Info zum heutigen informellen Gespräch"} --> hat ne Mail ans Rektorat geschrieben, dass Praktikums-Kategorie auch reingehört, yey
\end{itemize}

\subsection{HCI Shop (APV)}
\begin{itemize}
    \item ???????? @Connor, bitte ergänz hier was, ich hab da nicht so gut zugehört, wegen der Praktikakostensache...
    \item
\end{itemize}

\subsection{VAC}
\begin{itemize}
    \item nicht so viel besprochen, da VAC ja nicht da war
    \item zu "More streamlined aptitude colloquium" meinte Jeschke, es gehöre zum PHD-Dasein dazu, dass man auch lerne, sich durch viele Dokumente und Reglementarien durchzuquälen. Auch so von wegen Ruf der ETH, es ginge nicht, dass ETH-PHDs wo anders an Unis kämen und noch nicht solche Erfahrungen gemacht hätten
    \coperet meinte, es wäre im LAC klar geregelt
    \item bei "PHD streamlining" bzw Terminfindung von Verteidigungen wird mal in AB drüber gesprochen, aber ist organisatorisch eh schon viel aufwand für die eine Person (Name vergessen), die das macht
    \item Hoffnung besteht, dass vllt es mit mehr digitalisierung auch coole tools gibt, die das flexibler und übersichtlicher machen

\end{itemize}


\end{document}