\documentclass{article}
\usepackage{graphicx} % Required for inserting images
%Welcome :)

\documentclass{article}

% Basic document formatting
\usepackage[utf8]{inputenc}     % Input encoding
\usepackage[T1]{fontenc}        % Font encoding
\usepackage{lmodern}            % Modern LaTeX fonts
\usepackage{geometry}           % Set page margins
\geometry{a4paper, total={170mm,257mm}, left=20mm, top=20mm} 
\usepackage{float}              % Handling of floating elements
\usepackage{fancyhdr}           % Fancy headers
\usepackage{lastpage}           % use \pageref{LastPage} to make page x of y footers
\setlength{\parindent}{0pt}     % No \noindent

% Figures
\usepackage{graphicx}           % For including images
\usepackage{caption}            % Using the caption package
\usepackage{wrapfig}            % For including Wrap Figures
\usepackage{subcaption}         % For subfigures within a figure environment
\usepackage{pgfplots}           % Drawing plots
\usepackage{pgf-pie}            % For creating pie charts
\captionsetup[figure]{labelfont=bf}
\captionsetup[table]{labelfont=bf}
\usepackage{asymptote}          % Zum Zeichnen verschiedener Plots 
\usepackage{pdfpages}           % Zum Einfügen ganzer PDFs

% Colorboxes
\usepackage[skins]{tcolorbox}   % Color Boxes

% Tables and long tables
\usepackage{tabularx}           % Advanced table features
\usepackage{longtable}          % For tables that span multiple pages
\usepackage{multirow}           % Allows for multirow cells in tables
\usepackage{booktabs}           % For professional-quality tables

% Math packages
\usepackage{amsmath}            % Enhanced mathematical formatting
\usepackage{amssymb}            % Extended symbol collection
\usepackage{amsfonts}           % Mathematical fonts
\usepackage[version=4]{mhchem}  % Chemische Formeln
\usepackage{mathtools}          % Mathematical tools to supplement amsmath
\numberwithin{equation}{section} % Numbers Equations with chapters
\usepackage{siunitx}            % Makes SI-Units

% Code display
\usepackage{listings}           % For displaying code
\usepackage{xcolor}             % For coloring code
\lstdefinestyle{mystyle}{
    backgroundcolor=\color{backcolour},   
    commentstyle=\color{codegreen},
    keywordstyle=\color{ao},
    numberstyle=\tiny\color{codegray},
    basicstyle=\ttfamily\footnotesize,
    breakatwhitespace=false,         
    breaklines=true,                 
    captionpos=b,                    
    keepspaces=true,                 
    numbers=left,                    
    numbersep=5pt,                  
    showspaces=false,                
    showstringspaces=false,
    showtabs=false,                  
    tabsize=2
}
\lstset{style=mystyle}

% Custom Colours
\definecolor{LightCyan}{rgb}{0.88,1,1}
\definecolor{dkgreen}{rgb}{0,0.6,0}
\definecolor{gray}{rgb}{0.5,0.5,0.5}
\definecolor{mauve}{rgb}{0.58,0,0.82}
\definecolor{codegreen}{rgb}{0,0.6,0}
\definecolor{codegray}{rgb}{0.5,0.5,0.5}
\definecolor{ao}{rgb}{0.0, 0.0, 1.0}
\definecolor{backcolour}{rgb}{0.95,0.95,0.92}

% Referencing
\usepackage[style=numeric, backend=biber, sorting=none]{biblatex} % Imports biblatex package
\addbibresource{MAIN.bib} %Import the bibliography file
\DeclareFieldFormat{labelnumberwidth}{\mkbibbrackets{#1}} %ensure that the label numbers in the bibliography are enclosed in brackets.
\usepackage{xurl}

% Hyperlinks in the document
\usepackage{hyperref}           % For adding hyperlinks


\begin{document}

\vspace{-4cm}
\includegraphics[width=0.25\linewidth]{Graphics/VCS-Logo.png}

\vspace{-1.5cm}
\begin{flushright}
\parbox[r]{5cm}{Zurich, \today}
\end{flushright}

\vspace{1.5cm}
\begin{center}
\textbf{\LARGE{Informelles Departmentsgespräch (D-CHAB)}} \\
\vspace{0.5cm}
\Large
VCS - Vereinigung der Studierenden der Chemie-, Biochemie – Chemische Biologie, 
Chemieingenieurwissenschaften und interdisziplinären Naturwissenschaften
\end{center}

\section{Finanzen}

\subsection{Druck Null Exsi - Fachvereinszeitschrift}
Der Null Exsi ist eine Sonderausgabe unserer Fachvereinszeitschrift Exsikkator, der allen neu eintretenden Studierenden am Ersti-Tag mitgegeben wird. Dort sind viele nützliche Infos rund ums Studium, Leben in Zürich und der Schweiz, Tipps zu den Vorlesungen und der Organisation der ETH enthalten. Der Null Exsi ist eine immer weiter steigende grosse Ausgabe des Budgetpostens und gerade deswegen weil wir ab nächstem Semester diesen auch an MoEBs ausgeben wollen, würden wir gerne erfragen, ob sich das D-CHAB an der Finanzierung dessen beteiligt. 
\begin{itemize}
    \item Auch für APV ersties drucekn? (MOEBS + 10 ca.)
    \item über ZKB-Konto kostenlos drucken???
\end{itemize}

\subsection{Marco Ponts Award}
Das Konzept des Marco Ponts Teaching Awards wurde bereits vorgestellt und befindet sich in der ersten Umsetzungsphase. Die Umsetzung des Preises war im Konzept nicht festgelegt, aber nach dem Austausch mit anderen Fachvereinen, die einen Teaching Award organisieren, haben wir uns darauf festgelegt, dass wir für die Preisträger:innen einen physischen und finanziellen Preis bereitstellen möchten. \\

Bei \textbf{einem Preistragendem} wird ein Preisgeld in Höhe von \textbf{500 CHF} zusammen mit dem physischen Preis und dem Gratulationsschreiben verliehen.\\

Bei \textbf{zwei Preisträger:innen} wird ein Preisgeld in Höhe von \textbf{300 CHF pro Person} zusammen mit dem physischen Preis und dem Gratulationsschreiben verliehen.\\

Bei \textbf{drei Preisträger:innen} wird ein Preisgeld in Höhe von \textbf{200 CHF pro Person} zusammen mit dem physischen Preis und dem Gratulationsschreiben verliehen.\\

Die Finanzierung des physischen Preises wird gerne durch die VCS und den VAC organisiert. Bei dem Preisgeld möchten wir gerne auf das D-CHAB zukommen und um die Finanzierung bitten. Neben der Auszeichnung der exzellenten Arbeit sehen wir einen weiteren grossen Vorteil in dem Teaching Award. Die Qualität der Lehre nimmt durch eine Auszeichnung stetig zu und motiviert Teaching Assistants in ihrer Arbeit. Das kommt auch dem D-CHAB zu Gute, da die Lehre und Qualität der Leistung der Studierenden stark durch den Übungsbetrieb beeinflusst wird.

\section{PAKETH}
Die immer näher rückende Umsetzung von PAKETH sorgt bei Studierenden für Fragen über die konkrete Umsetzung am D-CHAB.

\subsection{Buddy-System für Erstis}
Des Öfteren kam das Gespräch hierbei schon auf möglicherweise neue Unterstützungsangebote, die für die Studierenden geschaffen werden könnten, um sich im stärker komprimierten Semester schneller und besser zurechtzufinden. Ein Vorschlag unsererseits wäre es, eine Art Buddy-System einzuführen, bei dem jüngeren Studierenden/ Ersties Buddies aus höheren Semestern als Ansprechpersonen für fachliche und organisatorische Fragen zugewiesen werden.\\
Die Administration eines solchen Programms vom D-CHAB aus würde unserer Ansicht nach am meisten Sinn ergeben, da unter PAKETH die Auslastung der Fachvereine bereits signifikant erhöht werden wird, sodass zusätzliche Organisation von Programmen kaum von Seiten der VCS tragbar sein würde. Für dies bitten wir um Ihre Einschätzung und ggf. Unterstützung.

\subsection{Studiengangsvernetzungstreffen}
Vorstellbar wäre darüber hinaus, angelehnt an den jedes Semester stattfindenden N-Treff auch ähnliche Vernetzungstreffen für alle anderen Studiengänge separat zu veranstalten. Unserer Ansicht nach könnten solche Studiengangs-Treffen zusammen mit einem Buddy-System einen grossen, positiven Einfluss auf die Involviertheit und die Leistungen der Studierenden haben.

\subsection{Notenbekanntgabe unter PAKETH}
Des Weiteren möchten wir um Auskunft bitten, wie der Bekanngabezeitraum der Noten unter PAKETH am D-CHAB geplant wird. Insbesondere mit nur der einen Woche Ferien zwischen Wintersession und Frühjahrssemester haben wir Bedenken, dass eine spätere Notenbekanntgabe im Frühjahrssemester den prompten Start in die Vorlesungen, mit der Möglichkeit zur Prüfungsabmeldung lediglich bis inklusive Woche 3, stark verschieben könnte. Gleichzeitig sind wir uns der hohen Aufwände der Dozierenden und Administrationen in diesem engen Zeitplan bewusst und schätzen deren Anstrengungen sehr.

\subsection{Workload Analyse}
Im Rahmen der Auswertung des Arbeits-Aufwands haben wir Daten als Teil des Semesterfeedbacks im HS24 und FS25 erhoben. Dort haben Studierende angegeben, wie viele Stunden pro Woche sie für die Bearbeitung ihrer Fächer benötigten. Diese Daten möchten wir dem Departement gerne zur verfügung stellen.

\subsection{Video-Aufzeichnungen}
Um während des Semesters bei den Lehrveranstaltungen möglichst gut dabeizubleiben, empfinden es viele Studierende als große Hilfe, wenn Sie Vorlesungen nachschauen können, insofern diese aufgezeichnet werden.\\
Zudem kommt es insbesondere bei Studierenden der Interdisziplinären Naturwissenschaften (N-lern) sehr häufig, sowie bei anderen Studiengängen gelgentlich zu Fächerüberschneidungen, bei welchen Videoaufzeichnungen wesentlich für das ausführliche Nacharbeiten der Fächer sind.\\ 
Ebenso trifft dies auf den Krankheitsfall zu - auch mit Hinblick auf die mehr in den Fokus rückende mentale Gesundhet der Studierenden kann es nur von Vorteil sein, verlässlich zu wissen, dass Vorlesungen jederzeit mit Video-Aufzeichnung nachgearbeitet werden können. \\ 
Daher möchten wir mit Hinblick auf PAKETH anbringen, dass wir ein verpflichtendes Aufzeichnen aller Vorlesungen sehr begrüssen würden. \\

\begin{itemize}
    \item symmetrischere semester (shcon bei immunologie, pharmazie beim APV, mehrere fächer mit prinzipieller reserve-möglichkeit), verschlanken des curriculums
    \item creditpunkteverteilung (wollen wir das aus dem walk\&talk kopieren?)
    \item Götti-programm: APV wäre auch interessiert (haben bisher ähnliches, aber auf eigeninitiative)\\
    erweiterung: Ansprechpersonen werden von uns gestellt, matching wird aber vom Department gemacht
\end{itemize}

\section{Anderes}
Kosten Praktikum: Offenlegen auf Website bei Studienkosten
---> APV: sehr variabel
glasswaren: Mittelwert vom letzten jahr
APV: analytik Prof steuer stellt kosten vor
(mehr sponsoring) (macht APV -> 15-60 chf übrig)
Laborkasten: Tauschbörse


APV Punkte:
-LSM holen gehen: Leute extrem unfreundlich (Prof. beschwert)
Mülltrennung (Glass) viel geändert

was stört abfall an praktikumsleuten
regulatorien besser kommunizieren
--> muss beim TA ankommen
Gruber? verantwortlicher für labor security ??

VAC: TA hours
-600-720h lps imps --> auch am abschluss transcript of teaching
-regelmässiges überprüfen, wie viel wird wirklich gemacht/ gebraucht (mit digital campus von paketh, per surveys von eth aus)

\subsection{HCI Shop}


\end{document}