\documentclass[a4paper]{article}

\usepackage[T1]{fontenc}
\usepackage[main=ngerman]{babel}
\usepackage[babel=true,tracking=true,kerning=true,letterspace=500,expansion=true,protrusion=true]{microtype}
\usepackage[german=swiss]{csquotes}
\MakeOuterQuote{"}

\usepackage{graphicx}
\usepackage{grffile}
\usepackage{enumitem}
\usepackage{libertinus}
\usepackage{geometry}
\geometry{
  a4paper,
  top=25mm,
  left=20mm,
  right=20mm,
  bottom=30mm,
  headsep=12mm,
  footskip=25mm,
  nomarginpar
}
\setlength\parindent{0pt}

\usepackage[bf,sf]{titlesec}
\usepackage{hyperref}
\usepackage{xurl}
\usepackage[table]{xcolor}
\usepackage{longtable}
\usepackage{comment}

\hypersetup{
    colorlinks=true,
    linkcolor=black,
    urlcolor=black
}
\urlstyle{rm}

% --- Dokumenttitel ---
\title{\vspace{-1em}\textbf{Konkrete Umsetzung von PAKETH am D-CHAB}}
\author{Connor Pütz (Präsident) \quad | \quad Paul Gärtner (HoPo-C) \quad | \quad Simon Gläser (HoPo-N)}
\date{\today}

\begin{document}
\maketitle

\tableofcontents
\newpage

% ================================================================
\section{Einleitung}

Das Projekt \textbf{PAKETH} (Prüfungen und Akademischer Kalender an der ETH Zürich) ist eine umfassende Lehr- und Studienreform, mit der die ETH Zürich ihre Studienstrukturen, Prüfungszyklen und Lehrkonzepte modernisieren möchte.  
Ziel des Projekts ist es, den akademischen Kalender zu vereinheitlichen, die Prüfungsphasen zu entzerren, Lehr- und Lernphasen klarer zu strukturieren und die Studierbarkeit der ETH-Programme zu verbessern. Dabei sollen sowohl Studierende als auch Lehrende von einer besseren Planbarkeit und Flexibilität profitieren.  

Im Zentrum von PAKETH stehen drei große Veränderungen:
\begin{itemize}[leftmargin=2em]
    \item eine \textbf{Neugestaltung des akademischen Kalenders}, um Prüfungs- und Lernphasen zu harmonisieren,
    \item die \textbf{Einführung modulbasierter Leistungsnachweise} anstelle großer Endprüfungen,
    \item und die \textbf{Optimierung der Lehr- und Lernbelastung} über das gesamte Semester hinweg.
\end{itemize}

Diese Reform bringt für das \textbf{Departement Chemie und Angewandte Biowissenschaften (D-CHAB)} jedoch besondere Herausforderungen mit sich.  
Das D-CHAB zeichnet sich durch eine intensive, praxisorientierte Ausbildung aus, die von zahlreichen Laborpraktika begleitet wird.  
Während andere Departemente vergleichsweise mehr reine Vorlesungszeit haben, ist der Anteil an verpflichtenden Praktika am D-CHAB deutlich höher.  

Die Umstellung des akademischen Kalenders im Rahmen von PAKETH bedeutet, dass die effektive Lernzeit zwischen den Unterrichtsphasen spürbar kürzer wird.  
Da die Praktika weiterhin einen großen Teil der wöchentlichen Arbeitszeit beanspruchen, entsteht für die Studierenden ein \textbf{sehr volles Semester}, in dem weniger Zeit zum selbstständigen Lernen, Wiederholen und Vertiefen bleibt.  

Diese erhöhte Belastung hat auch soziale und strukturelle Konsequenzen:  
Viele Studierende engagieren sich ehrenamtlich in studentischen Organisationen, Fachvereinen oder Kommissionen oder übernehmen als Teaching Assistants (TAs) wertvolle Aufgaben in der Lehre.  
Wenn durch die neue Struktur weniger zeitliche Freiräume bestehen, besteht die \textbf{Gefahr eines Rückgangs dieses Engagements}, was langfristig die studentische Mitgestaltung und die Qualität der Lehre beeinträchtigen könnte.  

Daher ist es entscheidend, dass PAKETH am D-CHAB nicht einfach nur organisatorisch umgesetzt, sondern inhaltlich \textbf{durchdacht und fachgerecht adaptiert} wird.  
Eine erfolgreiche Umsetzung muss darauf abzielen, die \textbf{Arbeitslast zu reduzieren}, ohne die \textbf{Qualität der Ausbildung zu gefährden}.  
Dies erfordert insbesondere:
\begin{itemize}[leftmargin=2em]
    \item eine bessere Abstimmung zwischen Vorlesungen, Übungen und Praktika,
    \item die inhaltliche Straffung von Lehrveranstaltungen ohne Substanzverlust,
    \item und eine klare Priorisierung der Lernziele in allen Lehrmodulen.
\end{itemize}

Ziel dieses Dokuments ist es, Vorschläge und konkrete Maßnahmen für die Umsetzung von PAKETH am D-CHAB darzulegen.  
Dabei soll der Fokus auf einer qualitativ hochwertigen, aber realistisch gestalteten Ausbildung liegen, die Studierende fordert, aber nicht überfordert, und die Raum für persönliches und akademisches Engagement lässt.

% ================================================================

\section{Erörterung der Workloadanalyse}

Lorem Ipsum

\section{Änderungen an den Curricula der einzelnen Studiengänge}

\subsection{Chemie BSc}

Die Anpassung des Chemie-Bachelorstudiengangs an PAKETH erfordert eine umfassende Neustrukturierung der Modulverteilung über alle sechs Semester. Die Herausforderung liegt darin, die hohe fachliche Qualität der Ausbildung zu erhalten, während gleichzeitig die Arbeitsbelastung für die Studierenden reduziert wird.

\textbf{Erstes Semester:} Das erste Semester erhält eine neue mathematische Grundlage durch die Verlegung der Linearen Algebra vom zweiten ins erste Semester. Diese Verschiebung ist essentiell, da die Lineare Algebra ohne Statistik als reine Tafelvorlesung deutlich ausführlicher behandelt werden kann und damit eine solidere Basis für spätere Module schafft. Die Analysis I wird um Differentialgleichungen als erstes Thema erweitert, um die Studierenden optimal auf die Physikalische Chemie vorzubereiten. Bei ACAC I werden die Sillen-Diagramme aus dem Curriculum entfernt, um Platz für eine umfassende Behandlung der MO-Theorie aus verschiedenen Perspektiven zu schaffen. Das Informatik-Modul wird gezielt darauf ausgerichtet, Grundlagen für fortgeschrittene Kurse wie Analytical and Physical Chemistry (APC), Digital Chemistry, Computer-aided Drug Design und Molecular Dynamics zu legen. In ACOC I werden homo- und isodesmische Reaktionen gestrichen, während Symmetrie-Inhalte nach ACAC II verlagert werden. Das AC-Praktikum wird gestrafft: Versuche wie der Ionentrennungsgang, überflüssige Titrationen und die Komplexsynthese ohne ausreichende theoretische Grundlagen werden reduziert, da diese Inhalte später in OACP II mit besserer Ausstattung und theoretischem Hintergrund behandelt werden.

\textbf{Zweites Semester:} ACAC II wird von redundanten Inhalten wie den Hundschen Regeln befreit. Die Analysis II erhält konzeptuelle Integralsätze (Stokes, Green, Gauss). Die Physikalische Chemie I wird praxisorientierter gestaltet mit mehr Phasendiagrammen und klassischem Rechnen bei gleichzeitiger Reduktion von Mikrotheorie und Herleitungen. In ACAC II wird der gesamte Hauptgruppenchemie-Teil entfernt und beim Kristallgitter-Teil bereits Inhalte von AC II vorweggenommen. Statt homodesmischer Reaktionen wird eine NMR-Einführung implementiert. OACP I wird um NMR-Inhalte erweitert, während zwei Experimente in der zweiten Semesterhälfte gestrichen werden. Die Biochemie wird ins zweite Semester verlegt, um das vierte Semester zu entlasten. Dabei werden Genetik und Kinetik entfernt und eventuell ein kleiner Teil zur bioanorganischen Chemie ergänzt. ACOC II bleibt als bewährte, runde Vorlesung erhalten.

\textbf{Drittes Semester:} OACP II wird teilweise in die Ferien verlegt, wobei die Hälfte der Stunden vor Semesterbeginn für Geräteeinführung und Sicherheitstests genutzt wird.

\textbf{Viertes Semester:} Das PPAC-Praktikum wird durch weniger Reports und mehr Präsentationen entlastet. Eine mögliche Umstrukturierung sieht vor, den Analytik-Teil im vierten Semester zu belassen und den PC-Teil ins fünfte Semester mit Teilen des Spektroskopie-Labors zu verlagern, während der Rest des Spektroskopie-Labors ins sechste Semester wandert.

\textbf{Fünftes Semester:} Der EPR-Teil von PC IV wird als eigenständige Vorlesung in den Master verlagert. OC III und OC IV sind qualitativ hochwertige Vorlesungen mit überschaubaren Konzepten, jedoch erfordern sie extensive Prüfungsvorbereitung durch den hohen Übungsaufwand. Dies kann durch vereinfachte Prüfungsformate gemildert werden.

\textbf{Sechstes Semester:} Das Sicherheitsmodul kann mit weniger ECTS-Punkten ausgestattet werden. AC IV könnte als Kernfach in den Master verlagert werden, was eine generelle Reform zu einem 120-ECTS-Master in vier Semestern unterstützen würde. Bei PC V kann der Lernaufwand durch besseres Erwartungsmanagement reduziert werden: Eine klare Unterscheidung zwischen prüfungsrelevanten Grundkonzepten und vertiefenden Erklärungen, die primär zum Nachschlagen dienen.

Die Neustrukturierung des Chemie-Bachelorstudiums unter PAKETH erfordert eine durchdachte Anpassung der Modulverteilung, um die besonderen Herausforderungen des Faches zu berücksichtigen.

Das zweite Studienjahr ist deutlich anspruchsvoller als das Basisjahr und erfordert eine solide mathematische Grundlage. Insbesondere für die Physikalische Chemie III (Molekulare Quantenmechanik) wird eine vertiefte Kenntnis der Linearen Algebra benötigt. Daher wird die Lineare Algebra bereits im ersten Semester platziert, um den Studierenden die nötigen mathematischen Werkzeuge frühzeitig zu vermitteln.

Die Biochemie wird strategisch ins zweite Semester verschoben, um das vierte Semester zu entlasten. Diese Maßnahme ist besonders wichtig, da das dritte und vierte Semester sehr zeitintensive Praktika beinhalten, die eine hohe Arbeitsbelastung für die Studierenden darstellen. Der Inhalt der Biochemie wird dabei gestrafft: sowohl der Reaktionskinetik- als auch der Genetik-Teil werden aus dem Curriculum gestrichen, um das Modul zu fokussieren und die Arbeitsbelastung zu reduzieren.

Eine weitere wichtige Änderung betrifft das Informatik-Modul: Es wird von einer Prüfung zu einer benoteten Semesterleistung umgestellt. Die Bewertung erfolgt über wöchentliche Abgaben von Übungsaufgaben, was eine kontinuierlichere Lernbetreuung ermöglicht und die Prüfungsbelastung in der Prüfungsphase reduziert.

\renewcommand{\arraystretch}{1.2}

\begin{longtable}{|p{0.64\textwidth}|>{\centering\arraybackslash}p{0.06\textwidth}|>{\centering\arraybackslash}p{0.09\textwidth}|>{\centering\arraybackslash}p{0.04\textwidth}|>{\centering\arraybackslash}p{0.04\textwidth}|}
\hline
\rowcolor{gray!60}
\textbf{PAKETH (Vorschlag)- 183 KP} & \textbf{Typ} & \textbf{PR} & \textbf{NG} & \textbf{KP} \\
\hline
\endfirsthead

\hline
\rowcolor{gray!60}
\textbf{PAKETH (Vorschlag) - 183 KP} & \textbf{Typ} & \textbf{PR} & \textbf{NG} & \textbf{KP} \\
\hline
\endhead

% ============================ Basisjahr ============================
\rowcolor{gray!40}
\multicolumn{5}{|l|}{\textbf{a. Module des Basisjahrs (Notengewichte) – 52 KP}} \\ \hline

\rowcolor{gray!20}
\multicolumn{5}{|l|}{\quad\textbf{Basisprüfungsgruppe A (Pflichtmodule mit Kompensation – 26 KP)}} \\ \hline
Allgemeine Chemie I (AC) & 2V+1U & 60\,s & 3 & \textcolor{red}{4} \\ \hline
Allgemeine Chemie I (OC) & 2V+1U & 60\,s & 3 & \textcolor{red}{4} \\ \hline
Allgemeine Chemie I (PC) & 2V+1U & 60\,s & 3 & \textcolor{red}{4} \\ \hline
Physik I & 3V+1U & 90\,s & 3 & 4 \\ \hline
Analysis I & 3V+2U & 60\,s & 3 & \textcolor{red}{5} \\ \hline
Informatik & 2V+2U & - & - & \textcolor{red}{5} \\ \hline

\rowcolor{gray!20}
\multicolumn{5}{|l|}{\quad\textbf{Basisprüfungsgruppe B (Pflichtmodule mit Kompensation – 26 KP)}} \\ \hline
Allgemeine Chemie II (AC) & 3V+1U & 60\,s & 3 & 4 \\ \hline
Allgemeine Chemie II (OC) & 3V+1U & 60\,s & 3 & \textcolor{red}{5} \\ \hline
Physikalische Chemie I: Thermodynamik & 3V+1U & 60\,s & 3 & 4 \\ \hline
Physik II & 3V+1U & 90\,s & 3 & 4 \\ \hline
Analysis II & 2V+1U & 60\,s & 3 & \textcolor{red}{4} \\ \hline
Lineare Algebra & 3V+2U & 120\,s & 3 & \textcolor{red}{5}  \\ \hline

% =================== Module höheres Bachelorstudium ===================
\rowcolor{gray!40}
\multicolumn{5}{|l|}{\textbf{b. Module höheres Bachelorstudium – 85 KP}} \\ \hline

\rowcolor{gray!20}
\multicolumn{5}{|l|}{\quad\textbf{Kernmodulgruppe A (Pflichtmodule mit Kompensation – 19 KP)}} \\ \hline
Anorganische Chemie I & 2V+1U & 90\,s & 3 & \textcolor{red}{4} \\ \hline
Organische Chemie I & 2V+1U & 60\,s & 3 & \textcolor{red}{4} \\ \hline
Physikalische Chemie II: Chemische Reaktionskinetik & 2V+1U & 90\,s & 3 & 4 \\ \hline
Analytische Chemie I & 3G & 60\,s & 3 & 3 \\ \hline
Analysis III: Partielle Differenzialgleichungen & 2V+1U & 120\,s & 2 & 4 \\ \hline

\rowcolor{gray!20}
\multicolumn{5}{|l|}{\quad\textbf{Kernmodulgruppe B (Pflichtmodule mit Kompensation – 23 KP)}} \\ \hline
Anorganische Chemie II & 2V+1U & 90\,s & 3 & \textcolor{red}{4} \\ \hline
Organische Chemie II & 2V+1U & 60\,s & 3 & \textcolor{red}{4} \\ \hline
Physikalische Chemie III: Molekulare Quantenmechanik & 3V+1U & 90\,s & 3 & \textcolor{red}{5} \\ \hline
Analytische Chemie II & 3G & 60\,s & 3 & 3 \\ \hline
Chemieingenieurwissenschaften & 2V+1U & 120\,s & 3 & \textcolor{red}{4} \\ \hline
Biochemie & 2G & 90\,s & 2 & \textcolor{red}{3} \\ \hline

\rowcolor{gray!20}
\multicolumn{5}{|l|}{\quad\textbf{Kernmodulgruppe C (Pflichtmodule mit Kompensation – 13 KP)}} \\ \hline
Anorganische Chemie III: Metallorganische Chemie und Homogenkatalyse & 3G & 60\,s\,+\,30\,m & 3 & 4 \\ \hline
Organische Chemie III: Einführung in die Asymmetrische Synthese & 3G & 60\,s\,+\,30\,m & 3 & 4 \\ \hline
Physikalische Chemie IV: Magnetische Resonanz & 3G & 30\,m & 3 & \textcolor{red}{5} \\ \hline

\rowcolor{gray!20}
\multicolumn{5}{|l|}{\quad\textbf{Kernmodulgruppe D (Pflichtmodule mit Kompensation – 14 KP)}} \\ \hline
Anorganische Chemie IV: Nanomaterialien: Synthese, Eigenschaften und Oberflächenchemie & 3G & 30\,m & 3 & 4 \\ \hline
Organische Chemie IV: Physikalisch Organische Chemie & 3G & 60\,s\,+\,30\,m & 3 & 4 \\ \hline
Physikalische Chemie V: Spektroskopie & 3G & 30\,m & 3 & 4 \\ \hline
Sicherheit & 2G & 180\,s & 2 & \textcolor{red}{2} \\ \hline

\rowcolor{gray!20}
\multicolumn{5}{|l|}{\quad\textbf{Vertiefungsmodule (Wahlpflichtmodule – 10 KP)}} \\ \hline
\textit{Gemäß Wahl der Studierenden} & - & - & - & 10 \\ \hline

\rowcolor{gray!20}
\multicolumn{5}{|l|}{\quad\textbf{Wissenschaft im Kontext (WIK) – Wahlpflichtmodule – 6 KP}} \\ \hline
\textit{Gemäß Vorgabe des D-CHAB} & - & - & - & 6 \\ \hline

\rowcolor{gray!35}
\multicolumn{5}{|l|}{\textbf{c. Praxismodule – Pflichtmodule – 46 KP}} \\ \hline
Allgemeine Chemie (Praktikum) & 10P & - & - & 7 \\ \hline
Anorganische und Organische Chemie I & 8P & - & - & 8 \\ \hline
Anorganische und Organische Chemie II & 16P & - & - & 11 \\ \hline
Analytische Chemie & 8P & - & - & 6 \\ \hline
Physikalische Chemie & 8P & - & - & 6 \\ \hline
Spektroskopie & 8P & - & - & 8 \\ \hline
\end{longtable}

\section{Alternative Leistungsnachweise}

Die Neugestaltung des Chemie-Bachelorstudiums unter PAKETH eröffnet Möglichkeiten für flexiblere und studierendenfreundlichere Prüfungsformen. Anstatt ausschließlich auf Endprüfungen zu setzen, können verschiedene Module durch alternative Leistungsnachweise bewertet werden, die eine kontinuierlichere Betreuung ermöglichen und die Prüfungsbelastung entzerren.

\subsection{Midterm-Prüfungen}

Folgende Module eignen sich besonders für eine Bewertung durch Midterm-Prüfungen, da sie aufbauenden Charakter haben und eine frühzeitige Leistungsrückmeldung sinnvoll ist:

\begin{itemize}
    \item \textbf{Allgemeine Chemie I (AC)} - Midterm nach 7 Wochen zur Überprüfung der Grundlagen
    \item \textbf{Allgemeine Chemie II (AC)} - Aufbauend auf ACAC I, kontinuierliche Wissenssicherung
    \item \textbf{Analysis I} - Mathematische Grundlagen werden schrittweise aufgebaut
    \item \textbf{Analysis II} - Vertiefung der mathematischen Konzepte mit Zwischenevaluation
    \item \textbf{Anorganische Chemie I} - Grundlegende AC-Konzepte mit Zwischenprüfung
    \item \textbf{Anorganische Chemie III} - Komplexere AC-Themen profitieren von geteilter Prüfungsbelastung
\end{itemize}

Die Midterm-Prüfungen finden idealerweise in der 8. Semesterwoche statt und decken etwa 40-50\% des Gesamtstoffs ab. Die Endprüfung fokussiert sich dann auf die verbleibenden Inhalte und Verknüpfungen zwischen den Themenbereichen.

\subsection{Notenbonus-Systeme}

Für Module mit starkem Übungscharakter oder praktischen Anteilen bieten sich Notenbonus-Systeme an, die kontinuierliche Mitarbeit und regelmäßige Leistung belohnen:

\begin{itemize}
    \item \textbf{Allgemeine Chemie I (OC)} - Wöchentliche Übungsabgaben mit Bonus zur Endnote
    \item \textbf{Allgemeine Chemie II (OC)} - Fortsetzung des Bonussystems für organische Grundlagen
    \item \textbf{Physik I} - Experimentelle Übungen und Hausaufgaben als Bonusleistung
    \item \textbf{Physik II} - Vertiefung mit praktischen Übungseinheiten und Bonuspunkten
    \item \textbf{Organische Chemie I} - Komplexe Reaktionsmechanismen durch kontinuierliche Übung
    \item \textbf{Organische Chemie II} - Aufbauende Synthesestrategien mit regelmäßiger Evaluation
    \item \textbf{Organische Chemie IV} - Spezialisierte OC-Themen mit vertiefenden Übungsaufgaben
\end{itemize}

Das Bonussystem kann bis zu 0.5 Notenpunkte Verbesserung zur Endprüfung beitragen, wenn mindestens 80\% der Übungsaufgaben erfolgreich bearbeitet wurden. Dies motiviert zur kontinuierlichen Mitarbeit und reduziert die Abhängigkeit von einer einzigen Prüfungsleistung.

\subsection{Vorteile der alternativen Bewertungsformen}

\begin{itemize}
    \item \textbf{Entzerrung der Prüfungsphasen:} Durch Midterms und kontinuierliche Bewertung wird die Belastung gleichmäßiger über das Semester verteilt
    \item \textbf{Bessere Lernbegleitung:} Frühzeitige Rückmeldungen ermöglichen rechtzeitige Korrekturen im Lernprozess
    \item \textbf{Reduzierte Prüfungsangst:} Multiple Bewertungsmöglichkeiten verringern den Druck auf einzelne Prüfungstermine
    \item \textbf{Förderung kontinuierlichen Lernens:} Bonussysteme motivieren zur regelmäßigen Auseinandersetzung mit dem Stoff
    \item \textbf{Praxisnähere Bewertung:} Alternative Formate spiegeln besser die spätere Berufspraxis wider
\end{itemize}

Die Implementation dieser alternativen Leistungsnachweise erfordert eine enge Koordination zwischen den Dozierenden und eine entsprechende Anpassung der Prüfungsordnung. Die gewonnene Flexibilität trägt jedoch erheblich zur Verbesserung der Studienbedingungen und zur Reduktion der Arbeitsbelastung bei.

\section{Conclusion}

Die dargestellte Struktur zeigt die neue modulare Organisation des Chemie-Bachelorstudiums unter PAKETH.  
Das Modell fasst verwandte Lehrveranstaltungen zu größeren Modulen zusammen und stärkt die Kohärenz zwischen Theorie und Praxis.  
Für die erfolgreiche Umsetzung sind jedoch gezielte Anpassungen bei Praktika, Prüfungszeitpunkten und Leistungsnachweisen erforderlich, um die Balance zwischen Workload und Qualität zu gewährleisten.

\vfill
\noindent\textbf{Kontakt:}\\
\href{mailto:puetzc@vcs.ethz.ch}{puetzc@vcs.ethz.ch} \quad
\href{mailto:pgaertner@vcs.ethz.ch}{pgaertner@vcs.ethz.ch} \quad
\href{mailto:glaesers@vcs.ethz.ch}{glaesers@vcs.ethz.ch}

\end{document}
