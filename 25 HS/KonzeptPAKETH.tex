\documentclass[a4paper]{article}

\usepackage[T1]{fontenc}
\usepackage[main=ngerman]{babel}
\usepackage[babel=true,tracking=true,kerning=true,letterspace=500,expansion=true,protrusion=true]{microtype}
\usepackage[german=swiss]{csquotes}
\MakeOuterQuote{"}

\usepackage{graphicx}
\usepackage{grffile}
\usepackage{enumitem}
\usepackage{libertinus}
\usepackage{geometry}
\geometry{
  a4paper,
  top=25mm,
  left=20mm,
  right=20mm,
  bottom=30mm,
  headsep=12mm,
  footskip=25mm,
  nomarginpar
}
\setlength\parindent{0pt}

\usepackage[bf,sf]{titlesec}
\usepackage{hyperref}
\usepackage{xurl}
\usepackage[table]{xcolor}
\usepackage{longtable}
\usepackage{comment}

\hypersetup{
    colorlinks=true,
    linkcolor=black,
    urlcolor=black
}
\urlstyle{rm}

% --- Dokumenttitel ---
\title{\vspace{-1em}\textbf{Konkrete Umsetzung von PAKETH am D-CHAB}}
\author{Connor Pütz (Präsident) \quad | \quad Paul Gärtner (HoPo-C) \quad | \quad Simon Gläser (HoPo-N)}
\date{\today}

\begin{document}
\maketitle

\tableofcontents
\newpage

% ================================================================
\section{Einleitung}

Das Projekt \textbf{PAKETH} (Prüfungen und Akademischer Kalender an der ETH Zürich) ist eine umfassende Lehr- und Studienreform, mit der die ETH Zürich ihre Studienstrukturen, Prüfungszyklen und Lehrkonzepte modernisieren möchte.  
Ziel des Projekts ist es, den akademischen Kalender zu vereinheitlichen, die Prüfungsphasen zu entzerren, Lehr- und Lernphasen klarer zu strukturieren und die Studierbarkeit der ETH-Programme zu verbessern. Dabei sollen sowohl Studierende als auch Lehrende von einer besseren Planbarkeit und Flexibilität profitieren.  

Im Zentrum von PAKETH stehen drei große Veränderungen:
\begin{itemize}[leftmargin=2em]
    \item eine \textbf{Neugestaltung des akademischen Kalenders}, um Prüfungs- und Lernphasen zu harmonisieren,
    \item die \textbf{Einführung modulbasierter Leistungsnachweise} anstelle großer Endprüfungen,
    \item und die \textbf{Optimierung der Lehr- und Lernbelastung} über das gesamte Semester hinweg.
\end{itemize}

Diese Reform bringt für das \textbf{Departement Chemie und Angewandte Biowissenschaften (D-CHAB)} jedoch besondere Herausforderungen mit sich.  
Das D-CHAB zeichnet sich durch eine intensive, praxisorientierte Ausbildung aus, die von zahlreichen Laborpraktika begleitet wird.  
Während andere Departemente vergleichsweise mehr reine Vorlesungszeit haben, ist der Anteil an verpflichtenden Praktika am D-CHAB deutlich höher.  

Die Umstellung des akademischen Kalenders im Rahmen von PAKETH bedeutet, dass die effektive Lernzeit zwischen den Unterrichtsphasen spürbar kürzer wird.  
Da die Praktika weiterhin einen großen Teil der wöchentlichen Arbeitszeit beanspruchen, entsteht für die Studierenden ein \textbf{sehr volles Semester}, in dem weniger Zeit zum selbstständigen Lernen, Wiederholen und Vertiefen bleibt.  

Diese erhöhte Belastung hat auch soziale und strukturelle Konsequenzen:  
Viele Studierende engagieren sich ehrenamtlich in studentischen Organisationen, Fachvereinen oder Kommissionen oder übernehmen als Teaching Assistants (TAs) wertvolle Aufgaben in der Lehre.  
Wenn durch die neue Struktur weniger zeitliche Freiräume bestehen, besteht die \textbf{Gefahr eines Rückgangs dieses Engagements}, was langfristig die studentische Mitgestaltung und die Qualität der Lehre beeinträchtigen könnte.  

Daher ist es entscheidend, dass PAKETH am D-CHAB nicht einfach nur organisatorisch umgesetzt, sondern inhaltlich \textbf{durchdacht und fachgerecht adaptiert} wird.  
Eine erfolgreiche Umsetzung muss darauf abzielen, die \textbf{Arbeitslast zu reduzieren}, ohne die \textbf{Qualität der Ausbildung zu gefährden}.  
Dies erfordert insbesondere:
\begin{itemize}[leftmargin=2em]
    \item eine bessere Abstimmung zwischen Vorlesungen, Übungen und Praktika,
    \item die inhaltliche Straffung von Lehrveranstaltungen ohne Substanzverlust,
    \item und eine klare Priorisierung der Lernziele in allen Lehrmodulen.
\end{itemize}

Ziel dieses Dokuments ist es, Vorschläge und konkrete Maßnahmen für die Umsetzung von PAKETH am D-CHAB darzulegen.  
Dabei soll der Fokus auf einer qualitativ hochwertigen, aber realistisch gestalteten Ausbildung liegen, die Studierende fordert, aber nicht überfordert, und die Raum für persönliches und akademisches Engagement lässt.

% ================================================================
\section{Änderungen an den Curricula der einzelnen Studiengänge}

\subsection{Chemie BSc}

%Hinzufügen warum das zweite jahr schwieriger ist

\begin{itemize}
    \item 1.Semester\\
    \item Lin Alg ins erste Semester und ohne Statistik (Tafelvorlesung ohne Matlab mehr Kredits, ausführlicher)siehe PC-III/IV gerade unnötig schweres Selbststudium
    \item Analysis: Erstes Thema DGL (für PC0)
    \item ACAC I: Sillen Diagramme raus nehmen
    \item Einmal vernünftig MO-Theorie, von allen perspektives
    \item Informatik I: soll Grundlagen für APC/Digital Chemistry/Computer-aided Drug Design/MD Courses schaffen (macht es das gerade???)
    \item ACOC I, braucht keine homo-/isodesmische Reaktionen; Symmetrie in ACAC II
    \item AC-Praktikum: Hier können einige Versuche reduziert werden, ohne die Lernziele des Kurses zu verfehlen. z.B. Ionentrennungsgang, weniger Titrationen, die "Komplexsynthese" hat keinen großen nutzen, da man die notwendigen Grundlagen nicht hat und man es eh rigoros und mit dem notwendigen Equipment in OACP II macht. Den einzigen Nutzen den dieser Versuch hat ist, das definitv jeder ersti den stechenden Geruch von ammoniak kennt.
    \item 2.Semester\\
    \item ACAC2: Hundsche Regeln brauch man nicht (nochmal)
    \item konzeptuell die Integralsätze(Stokes/Green/Gauss)in Analysis 2
    \item PCI: Mehr Phasendiagramme und klassisches Rechnen, weniger Mikro und Herleitungen
    \item Symmetrie in ACAC II; den ganzen Hauptgruppenchemie Teil von Grützmacher Raus; evtl. beim Kristallgitter teil ein paar sachen von AC II vorgreifen
    \item Statt homodesmische Reaktionen (lieber NMR-Einführung)
    \item OACPI mit NMR; 2 Experimente in der 2. Hälfte können easy gestrichen werden.
    \item Biochem für Chemiker in 2. Sem.
    \item Biochemie: Genetik und Kinetik rausnehmen; Evtl. kleiner Teil bzgl. bioinorganic chemistry von Prof. Mougel
    \item ACOC II: war prinzipiell immer eine sehr runde Vorlesung
    \item 3.Semester\\
    \item OACP II teilweise in Ferien (Hälfte der Stunden vor Semesterbeginn, z.B. Einführung in Geräte, Safety Test, etc.)
    \item 4.Semester\\
    \item PPAC weniger Reports, mehr Präsentationen; evtl. den Analytik Teil im 4. Semester lassen und den PC Teil ins 5. Semester mit Teilen vom Spektroskopie Lab. Dafür den Rest vom Spektroskopie Lab ins 6. Semester  
    \item 5. Semester\\
    \item PC IV: EPR-Teil als einzelne Vorlesung im Master
    \item OC III und OC IV: Sind sehr gute Vorlesungen, mit nicht mal so vielen Komplexen konzepten Jedoch braucht die Prüfungsvorbereitung sehr lange, da man für OC SEEEEHR viel üben muss. Das kann man teilweise durch einfachere Prüfungen lösen. 
    \item 6. Semester\\
    \item Safety: kann honestly weniger ECTS bekommen
    \item AC IV: Kann Kernfach im Master werden (\textbf{Generelle Reform: 120 ECTS Master in 4 Semester})
    \item PC V: Hier könnte Lernaufwand durch besseres Expectation-Management reduziert werden: Was ist ein Grundkonzept und definitv Prüfungsrelevant vs. Was ist eine in-depth Explaination und muss für die Prüfung evtl. nicht im Detail verstanden werden bzw. dient mehr zum Nachlesen.
\end{itemize}

Die Neustrukturierung des Chemie-Bachelorstudiums unter PAKETH erfordert eine durchdachte Anpassung der Modulverteilung, um die besonderen Herausforderungen des Faches zu berücksichtigen.

Das zweite Studienjahr ist deutlich anspruchsvoller als das Basisjahr und erfordert eine solide mathematische Grundlage. Insbesondere für die Physikalische Chemie III (Molekulare Quantenmechanik) wird eine vertiefte Kenntnis der Linearen Algebra benötigt. Daher wird die Lineare Algebra bereits im ersten Semester platziert, um den Studierenden die nötigen mathematischen Werkzeuge frühzeitig zu vermitteln.

Die Biochemie wird strategisch ins zweite Semester verschoben, um das vierte Semester zu entlasten. Diese Maßnahme ist besonders wichtig, da das dritte und vierte Semester sehr zeitintensive Praktika beinhalten, die eine hohe Arbeitsbelastung für die Studierenden darstellen. Der Inhalt der Biochemie wird dabei gestrafft: sowohl der Reaktionskinetik- als auch der Genetik-Teil werden aus dem Curriculum gestrichen, um das Modul zu fokussieren und die Arbeitsbelastung zu reduzieren.

Eine weitere wichtige Änderung betrifft das Informatik-Modul: Es wird von einer Prüfung zu einer benoteten Semesterleistung umgestellt. Die Bewertung erfolgt über wöchentliche Abgaben von Übungsaufgaben, was eine kontinuierlichere Lernbetreuung ermöglicht und die Prüfungsbelastung in der Prüfungsphase reduziert.

\renewcommand{\arraystretch}{1.2}

\begin{longtable}{|p{0.7\textwidth}|>{\centering\arraybackslash}p{0.06\textwidth}|>{\centering\arraybackslash}p{0.06\textwidth}|>{\centering\arraybackslash}p{0.04\textwidth}|>{\centering\arraybackslash}p{0.04\textwidth}|}
\hline
\rowcolor{gray!60}
\textbf{PAKETH (Vorschlag)} & \textbf{Typ} & \textbf{PR} & \textbf{NG} & \textbf{KP} \\
\hline
\endfirsthead

\hline
\rowcolor{gray!60}
\textbf{PAKETH (Vorschlag)} & \textbf{Typ} & \textbf{PR} & \textbf{NG} & \textbf{KP} \\
\hline
\endhead

% ============================ Basisjahr ============================
\rowcolor{gray!40}
\multicolumn{5}{|l|}{\textbf{a. Module des Basisjahrs (Notengewichte) – 44 KP}} \\ \hline

\rowcolor{gray!20}
\multicolumn{5}{|l|}{\quad\textbf{Basisprüfungsgruppe A (Pflichtmodule mit Kompensation – 20 KP)}} \\ \hline
Allgemeine Chemie I (AC) & 2V+1U & 60\,s & 3 & \textcolor{red}{4} \\ \hline
Allgemeine Chemie I (OC) & 2V+1U & 60\,s & 3 & \textcolor{red}{5} \\ \hline
Allgemeine Chemie I (PC) & 2V+1U & 60\,s & 3 & \textcolor{red}{4} \\ \hline
Physik I & 3V+1U & 60\,s & 3 & \textcolor{green}{4} \\ \hline
Analysis I & 3V+2U & 60\,s & 3 & \textcolor{red}{5} \\ \hline
Lineare Algebra \textit{(neues LinAlg $\approx$ Analysis A)} & 2V+1U & 60\,s & 2 & \textcolor{red}{5}  \\ \hline
Informatik I & 2V+2U & 60\,s & 2 & \textcolor{red}{6} \\ \hline

\rowcolor{gray!20}
\multicolumn{5}{|l|}{\quad\textbf{Basisprüfungsgruppe B (Pflichtmodule mit Kompensation – 23 KP)}} \\ \hline
Allgemeine Chemie II (AC) \textit{(unser Vorschlag beinhaltet aber eine Erweiterung der ACAC II Inhalte)}& 3V+1U & 60\,s & 3 & \textcolor{red}{3} \\ \hline
Allgemeine Chemie II (OC) & 3V+1U & 60\,s & 3 & \textcolor{red}{4} \\ \hline
Physikalische Chemie I: Thermodynamik & 3V+1U & 60\,s & 3 & \textcolor{red}{5} \\ \hline
Physik II & 3V+1U & 60\,s & 3 & \textcolor{green}{4} \\ \hline
Analysis II & 2V+1U & 60\,s & 3 & \textcolor{green}{3} \\ \hline
Biochemie & 4G & 60\,s & 3 & \textcolor{red}{3} \\ \hline

% =================== Module höheres Bachelorstudium ===================
\rowcolor{gray!40}
\multicolumn{5}{|l|}{\textbf{b. Module höheres Bachelorstudium – 96 KP}} \\ \hline

\rowcolor{gray!20}
\multicolumn{5}{|l|}{\quad\textbf{Kernmodulgruppe A (Pflichtmodule mit Kompensation – 17 KP)}} \\ \hline
Anorganische Chemie I & 2V+1U & 60\,s & 3 & \textcolor{green}{2-3} \\ \hline
Organische Chemie I & 2V+1U & 60\,s & 4 & \textcolor{red}{4} \\ \hline
Physikalische Chemie II: Chemische Reaktionskinetik & 2V+1U & 60\,s & 3 & \textcolor{red}{2} \\ \hline
Analytische Chemie I & 2V+1U & 60\,s & 4 & \textcolor{green}{3} \\ \hline
Analysis III: Partielle Differenzialgleichungen & 2V+1U & 60\,s & 2 & \textcolor{red}{3} \\ \hline

\rowcolor{gray!20}
\multicolumn{5}{|l|}{\quad\textbf{Kernmodulgruppe B (Pflichtmodule mit Kompensation – 17 KP)}} \\ \hline
Anorganische Chemie II & 2V+1U & 60\,s & 3 & \textcolor{red}{4} \\ \hline
Organische Chemie II & 2V+1U & 60\,s & 4 & \textcolor{red}{4} \\ \hline
Physikalische Chemie III: Molekulare Quantenmechanik & 2V+1U & 60\,s & 3 & \textcolor{green}{4} \\ \hline
Analytische Chemie II & 2V+2U & 60\,s & 4 & \textcolor{green}{3} (inkl. Spektrenübung) \\ \hline
Chemieingenieurwissenschaften & 2V+1U & 60\,s & 3 & \textcolor{green}{3} \\ \hline

\rowcolor{gray!20}
\multicolumn{5}{|l|}{\quad\textbf{Kernmodulgruppe C (Pflichtmodule mit Kompensation – 12 KP)}} \\ \hline
Anorganische Chemie III: Metallorganische Chemie und Homogenkatalyse & 2V+1U & 30\,m & 3 & 3 \\ \hline
Organische Chemie III: Einführung in die Asymmetrische Synthese & 2V+1U & 30\,m & 3 & 3 \\ \hline
Physikalische Chemie IV: Magnetische Resonanz & 2V+1U & 30\,m & 3 & 3 \\ \hline

\rowcolor{gray!20}
\multicolumn{5}{|l|}{\quad\textbf{Kernmodulgruppe D (Pflichtmodule mit Kompensation – 16 KP)}} \\ \hline
Anorganische Chemie IV: Nanomaterialien: Synthese, Eigenschaften und Oberflächenchemie & 2V+1U & 30\,m & 3 & 3 \\ \hline
Organische Chemie IV: Physikalisch Organische Chemie & 2V+1U & 30\,m & 3 & 3 \\ \hline
Physikalische Chemie V: Spektroskopie & 2V+1U & 30\,m & 3 & 3 \\ \hline
Sicherheit & 2V+1U & 180\,s & 2 & 2 \\ \hline

\rowcolor{gray!20}
\multicolumn{5}{|l|}{\quad\textbf{Vertiefungsmodule (Wahlpflichtmodule – 15 KP)}} \\ \hline
\textit{Gemäß Wahl der Studierenden} & 2V+1U & variabel & - & 15 \\ \hline

\rowcolor{gray!20}
\multicolumn{5}{|l|}{\quad\textbf{Wissenschaft im Kontext (WIK) – Wahlpflichtmodule – 6 KP}} \\ \hline
\textit{Gemäß Vorgabe des D-CHAB} & 2V+1U & variabel & - & 6 \\ \hline

\rowcolor{gray!35}
\multicolumn{5}{|l|}{\textbf{c. Praxismodule – Pflichtmodule – 50 KP}} \\ \hline
Allgemeine Chemie (Praktikum) & 10P & - & - & 9 \\ \hline
Anorganische und Organische Chemie I & 8P & - & - & 8 \\ \hline
Anorganische und Organische Chemie II & 16P & - & - & 10 \\ \hline
Analytische Chemie & 8P & - & - & 5 \\ \hline
Physikalische Chemie & 8P & - & - & 5 \\ \hline
Spektroskopie & 8P & - & - & 13 \\ \hline
\end{longtable}

\begin{comment}
\subsection{Chemieingenieurwissenschaften BSc}

\renewcommand{\arraystretch}{1.2}

\begin{longtable}{|p{0.7\textwidth}|>{\centering\arraybackslash}p{0.06\textwidth}|>{\centering\arraybackslash}p{0.06\textwidth}|>{\centering\arraybackslash}p{0.04\textwidth}|>{\centering\arraybackslash}p{0.04\textwidth}|}
\hline
\rowcolor{gray!60}
\textbf{PAKETH (Vorschlag)} & \textbf{Typ} & \textbf{PR} & \textbf{NG} & \textbf{KP} \\
\hline
\endfirsthead

\hline
\rowcolor{gray!60}
\textbf{PAKETH (Vorschlag)} & \textbf{Typ} & \textbf{PR} & \textbf{NG} & \textbf{KP} \\
\hline
\endhead

\rowcolor{gray!20}
\multicolumn{5}{|l|}{\quad\textbf{Kernmodulgruppe C (Pflichtmodule mit Kompensation – 21 KP)}} \\ \hline
Thermodynamik für Chemieingenieure & 2V+1U & 60\,s & 3 & 3 \\ \hline
Stofftransport & 2V+1U & 60\,s & 3 & 3 \\ \hline
Wärmetransport und Strömungslehre & 2V+1U & 60\,s & 3 & 3 \\ \hline
Homogene Reaktionstechnik & 2V+1U & 60\,s & 3 & 3 \\ \hline
Mikrobiologie & 2V+1U & 60\,s & 3 & 3 \\ \hline
Statistische und Numerische Methoden & 2V+1U & 60\,s & 3 & 3 \\ \hline
Technologieunternehmertum & 2V+1U & 60\,s & 3 & 3 \\ \hline

\rowcolor{gray!20}
\multicolumn{5}{|l|}{\quad\textbf{Kernmodulgruppe D (Pflichtmodule mit Kompensation – 17 KP)}} \\ \hline
Industrielle Chemie & 2V+1U & 60\,s & 3 & 3 \\ \hline
Heterogene Reaktionstechnik & 2V+1U & 60\,s & 3 & 3 \\ \hline
Trennprozesstechnologie & 2V+1U & 60\,s & 3 & 3 \\ \hline
Regelungstechnik & 2V+1U & 60\,s & 3 & 3 \\ \hline
Chemometrik und Maschinelles Lernen & 2V+1U & 60\,s & 3 & 3 \\ \hline
Sicherheit & 2V+1U & - & 2 & 2 \\ \hline

\rowcolor{gray!20}
\multicolumn{5}{|l|}{\quad\textbf{Vertiefungsmodule (Wahlpflichtmodule – 15 KP)}} \\ \hline
\textit{Gemäß Wahl der Studierenden} & 2V+1U & variabel & - & 15 \\ \hline

\rowcolor{gray!20}
\multicolumn{5}{|l|}{\quad\textbf{Wissenschaft im Kontext (WIK) – Wahlpflichtmodule – 6 KP}} \\ \hline
\textit{Gemäß Vorgabe des D-CHAB} & 2V+1U & variabel & - & 6 \\ \hline

\rowcolor{gray!35}
\multicolumn{5}{|l|}{\textbf{c. Praxismodule – Pflichtmodule – 41 KP}} \\ \hline
\textit{Gemäß definierter Praktika} & 2V+1U & - & - & 41 \\ \hline
\end{longtable}
\end{comment}

\subsection{Biochemie BSc}

\renewcommand{\arraystretch}{1.2}

\begin{longtable}{|p{0.7\textwidth}|>{\centering\arraybackslash}p{0.06\textwidth}|>{\centering\arraybackslash}p{0.06\textwidth}|>{\centering\arraybackslash}p{0.04\textwidth}|>{\centering\arraybackslash}p{0.04\textwidth}|}
\hline
\rowcolor{gray!60}
\textbf{PAKETH (Vorschlag)} & \textbf{Typ} & \textbf{PR} & \textbf{NG} & \textbf{KP} \\
\hline
\endfirsthead

\hline
\rowcolor{gray!60}
\textbf{PAKETH (Vorschlag)} & \textbf{Typ} & \textbf{PR} & \textbf{NG} & \textbf{KP} \\
\hline
\endhead

% ============================ Basisjahr ============================
\rowcolor{gray!40}
\multicolumn{5}{|l|}{\textbf{a. Module des Basisjahrs (Notengewichte) – 43 KP}} \\ \hline

\rowcolor{gray!20}
\multicolumn{5}{|l|}{\quad\textbf{Basisprüfungsgruppe A (Pflichtmodule mit Kompensation – 20 KP)}} \\ \hline
Allgemeine Chemie I (AC) & 2V+1U & 60\,s & 3 & 3 \\ \hline
Allgemeine Chemie I (OC) & 2V+1U & 60\,s & 3 & 3 \\ \hline
Allgemeine Chemie I (PC) & 2V+1U & 60\,s & 3 & 3 \\ \hline
Physik I & 2V+1U & 60\,s & 3 & 3 \\ \hline
Analysis I & 2V+1U & 60\,s & 3 & 3 \\ \hline
Lineare Algebra & 2V+1U & 60\,s & 2 & 2 \\ \hline

\rowcolor{gray!20}
\multicolumn{5}{|l|}{\quad\textbf{Basisprüfungsgruppe B (Pflichtmodule mit Kompensation – 23 KP)}} \\ \hline
Allgemeine Chemie II (AC) & 2V+1U & 60\,s & 3 & 3 \\ \hline
Allgemeine Chemie II (OC) & 2V+1U & 60\,s & 3 & 3 \\ \hline
Physikalische Chemie I: Thermodynamik & 2V+1U & 60\,s & 3 & 3 \\ \hline
Physik II & 2V+1U & 60\,s & 3 & 3 \\ \hline
Analysis II & 2V+1U & 60\,s & 3 & 3 \\ \hline
Informatik I & 2V+1U & 60\,s & 2 & 2 \\ \hline
Biologie: Biochemie & 2V+1U & 60\,s & 3 & 3 \\ \hline

% =================== Module höheres Bachelorstudium ===================
\rowcolor{gray!40}
\multicolumn{5}{|l|}{\textbf{b. Module höheres Bachelorstudium – 114 KP}} \\ \hline

\rowcolor{gray!20}
\multicolumn{5}{|l|}{\quad\textbf{Kernmodulgruppe A (Pflichtmodule mit Kompensation – 24 KP)}} \\ \hline
Anorganische Chemie I & 2V+1U & 60\,s & 3 & 3 \\ \hline
Physikalische Chemie II & 2V+1U & 60\,s & 3 & 3 \\ \hline
Statistik II & 2V+1U & 60\,s & 2 & 2 \\ \hline
Informatik I & 2V+1U & 60\,s & 2 & 2 \\ \hline
Organische Chemie I & 2V+1U & 60\,s & 6 & 6 \\ \hline
Physik I & 2V+1U & 60\,s & 6 & 6 \\ \hline
Analytische Chemie I & 2V+1U & 60\,s & 6 & 6 \\ \hline

\rowcolor{gray!20}
\multicolumn{5}{|l|}{\quad\textbf{Kernmodulgruppe B (Pflichtmodule mit Kompensation – 20 KP)}} \\ \hline
Organische Chemie II & 2V+1U & 60\,s & 6 & 6 \\ \hline
Physik II & 2V+1U & 60\,s & 6 & 6 \\ \hline
Analytische Chemie II & 2V+1U & 60\,s & 6 & 6 \\ \hline
Biochemie & 2V+1U & 60\,s & 5 & 5 \\ \hline
Systembiologie & 2V+1U & 60\,s & 5 & 5 \\ \hline

\rowcolor{gray!20}
\multicolumn{5}{|l|}{\quad\textbf{Kernmodulgruppe C (Pflichtmodule mit Kompensation – 24 KP)}} \\ \hline
Molekular- und Strukturbiologie I & 2V+1U & 60\,s & 1 & 1 \\ \hline
Molekular- und Strukturbiologie II & 2V+1U & 60\,s & 1 & 1 \\ \hline
Nukleinsäuren und Kohlenhydrate & 2V+1U & 60\,s & 1 & 1 \\ \hline
Proteine und Lipide & 2V+1U & 60\,s & 1 & 1 \\ \hline
Organische Chemie für BCB & 2V+1U & 60\,s & 1 & 1 \\ \hline

\rowcolor{gray!30}
\multicolumn{5}{|l|}{\quad\textbf{Vertiefungsmodule – Wahlpflichtmodule – 46 KP}} \\ \hline

\rowcolor{gray!20}
\multicolumn{5}{|l|}{\qquad\textbf{Blockkurse – Wahlpflichtmodule – 24 KP}} \\ \hline
\textit{Gemäß Wahl der Studierenden} & 2V+1U & variabel & - & 24 \\ \hline

\rowcolor{gray!20}
\multicolumn{5}{|l|}{\qquad\textbf{Wahlmodule BCB – Wahlpflichtmodule – 16 KP}} \\ \hline
\textit{Gemäß Wahl der Studierenden} & 2V+1U & variabel & - & 16 \\ \hline

\rowcolor{gray!20}
\multicolumn{5}{|l|}{\qquad\textbf{Wissenschaft im Kontext – Wahlpflichtmodule – 6 KP}} \\ \hline
\textit{Gemäß Vorgabe des D-CHAB} & 2V+1U & variabel & - & 6 \\ \hline

\rowcolor{gray!35}
\multicolumn{5}{|l|}{\textbf{c. Praxismodule – Pflichtmodule – 30 KP}} \\ \hline
\textit{Gemäß definierter Praktika} & 2V+1U & - & - & 30 \\ \hline
\end{longtable}

\subsection{Interdisziplinäre Naturwissenschaften BSc (Biochemisch-physikalische Fachrichtung)}

\renewcommand{\arraystretch}{1.2}

\begin{longtable}{|p{0.7\textwidth}|>{\centering\arraybackslash}p{0.06\textwidth}|>{\centering\arraybackslash}p{0.06\textwidth}|>{\centering\arraybackslash}p{0.04\textwidth}|>{\centering\arraybackslash}p{0.04\textwidth}|}
\hline
\rowcolor{gray!60}
\textbf{PAKETH (Vorschlag)} & \textbf{Typ} & \textbf{PR} & \textbf{NG} & \textbf{KP} \\
\hline
\endfirsthead

\hline
\rowcolor{gray!60}
\textbf{PAKETH (Vorschlag)} & \textbf{Typ} & \textbf{PR} & \textbf{NG} & \textbf{KP} \\
\hline
\endhead

% ============================ Basisjahr ============================
\rowcolor{gray!40}
\multicolumn{5}{|l|}{\textbf{a. Module des Basisjahrs (Notengewichte) – 43 KP}} \\ \hline

\rowcolor{gray!20}
\multicolumn{5}{|l|}{\quad\textbf{Basisprüfungsgruppe A (Pflichtmodule mit Kompensation – 20 KP)}} \\ \hline
Allgemeine Chemie I (AC) & 2V+1U & 60\,s & 3 & 3 \\ \hline
Allgemeine Chemie I (OC) & 2V+1U & 60\,s & 3 & 3 \\ \hline
Allgemeine Chemie I (PC) & 2V+1U & 60\,s & 3 & 3 \\ \hline
Physik I & 2V+1U & 60\,s & 3 & 3 \\ \hline
Analysis I & 2V+1U & 60\,s & 3 & 3 \\ \hline
Lineare Algebra & 2V+1U & 60\,s & 2 & 2 \\ \hline

\rowcolor{gray!20}
\multicolumn{5}{|l|}{\quad\textbf{Basisprüfungsgruppe B (Pflichtmodule mit Kompensation – 23 KP)}} \\ \hline
Allgemeine Chemie II (AC) & 2V+1U & 60\,s & 3 & 3 \\ \hline
Allgemeine Chemie II (OC) & 2V+1U & 60\,s & 3 & 3 \\ \hline
Physikalische Chemie I: Thermodynamik & 2V+1U & 60\,s & 3 & 3 \\ \hline
Physik II & 2V+1U & 60\,s & 3 & 3 \\ \hline
Analysis II & 2V+1U & 60\,s & 3 & 3 \\ \hline
Informatik I & 2V+1U & 60\,s & 2 & 2 \\ \hline
Biologie: Biochemie & 2V+1U & 60\,s & 3 & 3 \\ \hline

% =================== Module höheres Bachelorstudium ===================
\rowcolor{gray!40}
\multicolumn{5}{|l|}{\textbf{b. Module höheres Bachelorstudium – 96 KP}} \\ \hline

\rowcolor{gray!20}
\multicolumn{5}{|l|}{\quad\textbf{Kernmodulgruppe A (Pflichtmodule mit Kompensation – 17 KP)}} \\ \hline
Anorganische Chemie I & 2V+1U & 60\,s & 3 & 3 \\ \hline
Organische Chemie I & 2V+1U & 60\,s & 4 & 4 \\ \hline
Physikalische Chemie II: Chemische Reaktionskinetik & 2V+1U & 60\,s & 3 & 3 \\ \hline
Analytische Chemie I & 2V+1U & 60\,s & 4 & 4 \\ \hline
Analysis III: Partielle Differenzialgleichungen & 2V+1U & 60\,s & 2 & 2 \\ \hline

\rowcolor{gray!20}
\multicolumn{5}{|l|}{\quad\textbf{Kernmodulgruppe B (Pflichtmodule mit Kompensation – 17 KP)}} \\ \hline
Anorganische Chemie II & 2V+1U & 60\,s & 3 & 3 \\ \hline
Organische Chemie II & 2V+1U & 60\,s & 4 & 4 \\ \hline
Physikalische Chemie III: Molekulare Quantenmechanik & 2V+1U & 60\,s & 3 & 3 \\ \hline
Analytische Chemie II & 2V+1U & 60\,s & 4 & 4 \\ \hline
Chemieingenieurwissenschaften & 2V+1U & 60\,s & 3 & 3 \\ \hline

\rowcolor{gray!20}
\multicolumn{5}{|l|}{\quad\textbf{Kernmodulgruppe C (Pflichtmodule mit Kompensation – 9 KP)}} \\ \hline
Anorganische Chemie III: Metallorganische Chemie und Homogenkatalyse & 2V+1U & 30\,m & 3 & 3 \\ \hline
Organische Chemie III: Einführung in die asymmetrische Synthese & 2V+1U & 30\,m & 3 & 3 \\ \hline
Physikalische Chemie IV: Magnetische Resonanz & 2V+1U & 30\,m & 3 & 3 \\ \hline

\rowcolor{gray!20}
\multicolumn{5}{|l|}{\quad\textbf{Kernmodulgruppe D (Pflichtmodule mit Kompensation – 11 KP)}} \\ \hline
Anorganische Chemie IV: Nanomaterialien: Synthese, Eigenschaften und Oberflächenchemie & 2V+1U & 30\,m & 3 & 3 \\ \hline
Organische Chemie IV: Physikalisch Organische Chemie & 2V+1U & 30\,m & 3 & 3 \\ \hline
Physikalische Chemie V: Spektroskopie & 2V+1U & 30\,m & 3 & 3 \\ \hline
Sicherheit & 2V+1U & - & 2 & 2 \\ \hline

\rowcolor{gray!20}
\multicolumn{5}{|l|}{\quad\textbf{Vertiefungsmodule (Wahlpflichtmodule – 15 KP)}} \\ \hline
\textit{Gemäß Wahl der Studierenden} & 2V+1U & variabel & - & 15 \\ \hline

\rowcolor{gray!20}
\multicolumn{5}{|l|}{\quad\textbf{Wissenschaft im Kontext (WIK) – Wahlpflichtmodule – 6 KP}} \\ \hline
\textit{Gemäß Vorgabe des D-CHAB} & 2V+1U & variabel & - & 6 \\ \hline

\rowcolor{gray!35}
\multicolumn{5}{|l|}{\textbf{c. Praxismodule – Pflichtmodule – 41 KP}} \\ \hline
\textit{Gemäß definierter Praktika} & 2V+1U & - & - & 41 \\ \hline
\end{longtable}

\subsection{Interdisziplinäre Naturwissenschaften BSc (Physikalisch-chemische Fachrichtung)}

\renewcommand{\arraystretch}{1.2}

\begin{longtable}{|p{0.7\textwidth}|>{\centering\arraybackslash}p{0.06\textwidth}|>{\centering\arraybackslash}p{0.06\textwidth}|>{\centering\arraybackslash}p{0.04\textwidth}|>{\centering\arraybackslash}p{0.04\textwidth}|}
\hline
\rowcolor{gray!60}
\textbf{PAKETH (Vorschlag)} & \textbf{Typ} & \textbf{PR} & \textbf{NG} & \textbf{KP} \\
\hline
\endfirsthead

\hline
\rowcolor{gray!60}
\textbf{PAKETH (Vorschlag)} & \textbf{Typ} & \textbf{PR} & \textbf{NG} & \textbf{KP} \\
\hline
\endhead

% ============================ Basisjahr ============================
\rowcolor{gray!40}
\multicolumn{5}{|l|}{\textbf{a. Module des Basisjahrs (Notengewichte) – 43 KP}} \\ \hline

\rowcolor{gray!20}
\multicolumn{5}{|l|}{\quad\textbf{Basisprüfungsgruppe A (Pflichtmodule mit Kompensation – 20 KP)}} \\ \hline
Allgemeine Chemie I (AC) & 2V+1U & 60\,s & 3 & 3 \\ \hline
Allgemeine Chemie I (OC) & 2V+1U & 60\,s & 3 & 3 \\ \hline
Allgemeine Chemie I (PC) & 2V+1U & 60\,s & 3 & 3 \\ \hline
Physik I & 2V+1U & 60\,s & 3 & 3 \\ \hline
Analysis I & 2V+1U & 60\,s & 3 & 3 \\ \hline
Lineare Algebra & 2V+1U & 60\,s & 2 & 2 \\ \hline

\rowcolor{gray!20}
\multicolumn{5}{|l|}{\quad\textbf{Basisprüfungsgruppe B (Pflichtmodule mit Kompensation – 23 KP)}} \\ \hline
Allgemeine Chemie II (AC) & 2V+1U & 60\,s & 3 & 3 \\ \hline
Allgemeine Chemie II (OC) & 2V+1U & 60\,s & 3 & 3 \\ \hline
Physikalische Chemie I: Thermodynamik & 2V+1U & 60\,s & 3 & 3 \\ \hline
Physik II & 2V+1U & 60\,s & 3 & 3 \\ \hline
Analysis II & 2V+1U & 60\,s & 3 & 3 \\ \hline
Informatik I & 2V+1U & 60\,s & 2 & 2 \\ \hline
Biologie: Biochemie & 2V+1U & 60\,s & 3 & 3 \\ \hline

% =================== Module höheres Bachelorstudium ===================
\rowcolor{gray!40}
\multicolumn{5}{|l|}{\textbf{b. Module höheres Bachelorstudium – 96 KP}} \\ \hline

\rowcolor{gray!20}
\multicolumn{5}{|l|}{\quad\textbf{Kernmodulgruppe A (Pflichtmodule mit Kompensation – 17 KP)}} \\ \hline
Anorganische Chemie I & 2V+1U & 60\,s & 3 & 3 \\ \hline
Organische Chemie I & 2V+1U & 60\,s & 4 & 4 \\ \hline
Physikalische Chemie II: Chemische Reaktionskinetik & 2V+1U & 60\,s & 3 & 3 \\ \hline
Analytische Chemie I & 2V+1U & 60\,s & 4 & 4 \\ \hline
Analysis III: Partielle Differenzialgleichungen & 2V+1U & 60\,s & 2 & 2 \\ \hline

\rowcolor{gray!20}
\multicolumn{5}{|l|}{\quad\textbf{Kernmodulgruppe B (Pflichtmodule mit Kompensation – 17 KP)}} \\ \hline
Anorganische Chemie II & 2V+1U & 60\,s & 3 & 3 \\ \hline
Organische Chemie II & 2V+1U & 60\,s & 4 & 4 \\ \hline
Physikalische Chemie III: Molekulare Quantenmechanik & 2V+1U & 60\,s & 3 & 3 \\ \hline
Analytische Chemie II & 2V+1U & 60\,s & 4 & 4 \\ \hline
Chemieingenieurwissenschaften & 2V+1U & 60\,s & 3 & 3 \\ \hline

\rowcolor{gray!20}
\multicolumn{5}{|l|}{\quad\textbf{Kernmodulgruppe C (Pflichtmodule mit Kompensation – 9 KP)}} \\ \hline
Anorganische Chemie III: Metallorganische Chemie und Homogenkatalyse & 2V+1U & 30\,m & 3 & 3 \\ \hline
Organische Chemie III: Einführung in die asymmetrische Synthese & 2V+1U & 30\,m & 3 & 3 \\ \hline
Physikalische Chemie IV: Magnetische Resonanz & 2V+1U & 30\,m & 3 & 3 \\ \hline

\rowcolor{gray!20}
\multicolumn{5}{|l|}{\quad\textbf{Kernmodulgruppe D (Pflichtmodule mit Kompensation – 11 KP)}} \\ \hline
Anorganische Chemie IV: Nanomaterialien: Synthese, Eigenschaften und Oberflächenchemie & 2V+1U & 30\,m & 3 & 3 \\ \hline
Organische Chemie IV: Physikalisch Organische Chemie & 2V+1U & 30\,m & 3 & 3 \\ \hline
Physikalische Chemie V: Spektroskopie & 2V+1U & 30\,m & 3 & 3 \\ \hline
Sicherheit & 2V+1U & - & 2 & 2 \\ \hline

\rowcolor{gray!20}
\multicolumn{5}{|l|}{\quad\textbf{Vertiefungsmodule (Wahlpflichtmodule – 15 KP)}} \\ \hline
\textit{Gemäß Wahl der Studierenden} & 2V+1U & variabel & - & 15 \\ \hline

\rowcolor{gray!20}
\multicolumn{5}{|l|}{\quad\textbf{Wissenschaft im Kontext (WIK) – Wahlpflichtmodule – 6 KP}} \\ \hline
\textit{Gemäß Vorgabe des D-CHAB} & 2V+1U & variabel & - & 6 \\ \hline

\rowcolor{gray!35}
\multicolumn{5}{|l|}{\textbf{c. Praxismodule – Pflichtmodule – 41 KP}} \\ \hline
\textit{Gemäß definierter Praktika} & 2V+1U & - & - & 41 \\ \hline
\end{longtable}

\section{Arbeitsbelastung}

\subsection{Chemie BSc}

Die folgende Aufstellung zeigt die detaillierte Arbeitsbelastung für alle Module im Chemie-Bachelorstudiengang unter PAKETH.

\subsubsection{Basisjahr - Erste Semester}

\renewcommand{\arraystretch}{1.1}

\begin{longtable}{|p{0.35\textwidth}|p{0.1\textwidth}|p{0.15\textwidth}|p{0.15\textwidth}|p{0.15\textwidth}|p{0.1\textwidth}|}
\hline
\rowcolor{gray!30}
\textbf{Modul} & \textbf{KP} & \textbf{Kontaktzeit} & \textbf{Selbststudium} & \textbf{Prüfungsvorbereitung} & \textbf{Gesamtaufwand} \\
\hline
\endfirsthead

\hline
\rowcolor{gray!30}
\textbf{Modul} & \textbf{KP} & \textbf{Kontaktzeit} & \textbf{Selbststudium} & \textbf{Prüfungsvorbereitung} & \textbf{Gesamtaufwand} \\
\hline
\endhead

\textbf{Allgemeine Chemie I (AC)} & 3 & 42h & 45h & 15h & 90h \\
\hline
\textbf{Allgemeine Chemie I (OC)} & 3 & 42h & 45h & 15h & 90h \\
\hline
\textbf{Allgemeine Chemie I (PC)} & 3 & 42h & 45h & 15h & 90h \\
\hline
\textbf{Physik I} & 3 & 42h & 45h & 15h & 90h \\
\hline
\textbf{Analysis I} & 3 & 42h & 50h & 10h & 90h \\
\hline
\textbf{Lineare Algebra} & 2 & 28h & 35h & 7h & 60h \\
\hline
\textbf{Allgemeine Chemie II (AC)} & 3 & 42h & 45h & 15h & 90h \\
\hline
\textbf{Allgemeine Chemie II (OC)} & 3 & 42h & 45h & 15h & 90h \\
\hline
\textbf{Physikalische Chemie I} & 3 & 42h & 45h & 15h & 90h \\
\hline
\textbf{Physik II} & 3 & 42h & 45h & 15h & 90h \\
\hline
\textbf{Analysis II} & 3 & 42h & 50h & 10h & 90h \\
\hline
\textbf{Informatik I} & 2 & 28h & 40h & 5h & 60h \\
\hline
\textbf{Biologie: Biochemie} & 3 & 42h & 35h & 20h & 90h \\
\hline
\rowcolor{gray!20}
\textbf{Summe Basisjahr} & \textbf{37} & \textbf{504h} & \textbf{580h} & \textbf{192h} & \textbf{1110h} \\
\hline
\end{longtable}

\subsubsection{Kernmodule höheres Bachelorstudium}

\begin{longtable}{|p{0.35\textwidth}|p{0.1\textwidth}|p{0.15\textwidth}|p{0.15\textwidth}|p{0.15\textwidth}|p{0.1\textwidth}|}
\hline
\rowcolor{gray!30}
\textbf{Modul} & \textbf{KP} & \textbf{Kontaktzeit} & \textbf{Selbststudium} & \textbf{Prüfungsvorbereitung} & \textbf{Gesamtaufwand} \\
\hline
\endfirsthead

\hline
\rowcolor{gray!30}
\textbf{Modul} & \textbf{KP} & \textbf{Kontaktzeit} & \textbf{Selbststudium} & \textbf{Prüfungsvorbereitung} & \textbf{Gesamtaufwand} \\
\hline
\endhead

\textbf{Anorganische Chemie I} & 3 & 42h & 40h & 18h & 90h \\
\hline
\textbf{Organische Chemie I} & 4 & 56h & 55h & 20h & 120h \\
\hline
\textbf{Physikalische Chemie II} & 3 & 42h & 40h & 18h & 90h \\
\hline
\textbf{Analytische Chemie I} & 4 & 56h & 50h & 24h & 120h \\
\hline
\textbf{Analysis III} & 2 & 28h & 35h & 7h & 60h \\
\hline
\textbf{Anorganische Chemie II} & 3 & 42h & 40h & 18h & 90h \\
\hline
\textbf{Organische Chemie II} & 4 & 56h & 55h & 20h & 120h \\
\hline
\textbf{Physikalische Chemie III} & 3 & 42h & 40h & 18h & 90h \\
\hline
\textbf{Analytische Chemie II} & 4 & 56h & 50h & 24h & 120h \\
\hline
\textbf{Chemieingenieurwissenschaften} & 3 & 42h & 35h & 23h & 90h \\
\hline
\textbf{Anorganische Chemie III} & 3 & 42h & 35h & 23h & 90h \\
\hline
\textbf{Organische Chemie III} & 3 & 42h & 35h & 23h & 90h \\
\hline
\textbf{Physikalische Chemie IV} & 3 & 42h & 35h & 23h & 90h \\
\hline
\textbf{Anorganische Chemie IV} & 3 & 42h & 35h & 23h & 90h \\
\hline
\textbf{Organische Chemie IV} & 3 & 42h & 35h & 23h & 90h \\
\hline
\textbf{Physikalische Chemie V} & 3 & 42h & 35h & 23h & 90h \\
\hline
\textbf{Sicherheit} & 2 & 28h & 25h & 7h & 60h \\
\hline
\rowcolor{gray!20}
\textbf{Summe Kernmodule} & \textbf{50} & \textbf{714h} & \textbf{720h} & \textbf{313h} & \textbf{1500h} \\
\hline
\end{longtable}

\subsubsection{Wahlmodule und Praktika}

\begin{longtable}{|p{0.35\textwidth}|p{0.1\textwidth}|p{0.15\textwidth}|p{0.15\textwidth}|p{0.15\textwidth}|p{0.1\textwidth}|}
\hline
\rowcolor{gray!30}
\textbf{Modul} & \textbf{KP} & \textbf{Kontaktzeit} & \textbf{Selbststudium} & \textbf{Prüfungsvorbereitung} & \textbf{Gesamtaufwand} \\
\hline
\endfirsthead

\hline
\rowcolor{gray!30}
\textbf{Modul} & \textbf{KP} & \textbf{Kontaktzeit} & \textbf{Selbststudium} & \textbf{Prüfungsvorbereitung} & \textbf{Gesamtaufwand} \\
\hline
\endhead

\textbf{Vertiefungsmodule} & 15 & 150h & 250h & 50h & 450h \\
\hline
\textbf{Wissenschaft im Kontext} & 6 & 84h & 90h & 6h & 180h \\
\hline
\textbf{Allgemeines Chemie Praktikum} & 8 & 120h & 80h & 40h & 240h \\
\hline
\textbf{OACP I} & 12 & 180h & 120h & 60h & 360h \\
\hline
\textbf{OACP II} & 12 & 180h & 120h & 60h & 360h \\
\hline
\textbf{PPAC} & 8 & 120h & 80h & 40h & 240h \\
\hline
\textbf{Spektroskopie Praktikum} & 10 & 150h & 100h & 50h & 300h \\
\hline
\textbf{Bachelor-Arbeit} & 12 & 50h & 300h & 10h & 360h \\
\hline
\rowcolor{gray!20}
\textbf{Summe Wahlmodule/Praktika} & \textbf{83} & \textbf{1034h} & \textbf{1140h} & \textbf{316h} & \textbf{2490h} \\
\hline
\end{longtable}

\subsubsection{Gesamtübersicht Chemie BSc}

\begin{longtable}{|p{0.35\textwidth}|p{0.1\textwidth}|p{0.15\textwidth}|p{0.15\textwidth}|p{0.15\textwidth}|p{0.1\textwidth}|}
\hline
\rowcolor{gray!40}
\textbf{Studienbereich} & \textbf{KP} & \textbf{Kontaktzeit} & \textbf{Selbststudium} & \textbf{Prüfungsvorbereitung} & \textbf{Gesamtaufwand} \\
\hline
\endfirsthead

\hline
\rowcolor{gray!40}
\textbf{Studienbereich} & \textbf{KP} & \textbf{Kontaktzeit} & \textbf{Selbststudium} & \textbf{Prüfungsvorbereitung} & \textbf{Gesamtaufwand} \\
\hline
\endhead

\textbf{Basisjahr} & 37 & 504h & 580h & 192h & 1110h \\
\hline
\textbf{Kernmodule} & 50 & 714h & 720h & 313h & 1500h \\
\hline
\textbf{Wahlmodule/Praktika} & 83 & 1034h & 1140h & 316h & 2490h \\
\hline
\rowcolor{gray!60}
\textbf{TOTAL CHEMIE BSc} & \textbf{170} & \textbf{2252h} & \textbf{2440h} & \textbf{821h} & \textbf{5100h} \\
\hline
\end{longtable}

\normalsize
\renewcommand{\arraystretch}{1.2}



\section{Conclusion}

Die dargestellte Struktur zeigt die neue modulare Organisation des Chemie-Bachelorstudiums unter PAKETH.  
Das Modell fasst verwandte Lehrveranstaltungen zu größeren Modulen zusammen und stärkt die Kohärenz zwischen Theorie und Praxis.  
Für die erfolgreiche Umsetzung sind jedoch gezielte Anpassungen bei Praktika, Prüfungszeitpunkten und Leistungsnachweisen erforderlich, um die Balance zwischen Workload und Qualität zu gewährleisten.

\vfill
\noindent\textbf{Kontakt:}\\
\href{mailto:puetzc@vcs.ethz.ch}{puetzc@vcs.ethz.ch} \quad
\href{mailto:pgaertner@vcs.ethz.ch}{pgaertner@vcs.ethz.ch} \quad
\href{mailto:glaesers@vcs.ethz.ch}{glaesers@vcs.ethz.ch}

\end{document}
