\documentclass[a4paper]{article}

\usepackage[T1]{fontenc}
\usepackage[main=ngerman]{babel}
\usepackage[babel=true,tracking=true,kerning=true,letterspace=500,expansion=true,protrusion=true]{microtype}
\usepackage[german=swiss]{csquotes}
\MakeOuterQuote{"}

\usepackage{graphicx}
\usepackage{grffile}
\usepackage{enumitem}
\usepackage{libertinus}
\usepackage{geometry}
\geometry{
  a4paper,
  top=25mm,
  left=20mm,
  right=20mm,
  bottom=30mm,
  headsep=12mm,
  footskip=25mm,
  nomarginpar
}
\setlength\parindent{0pt}

\usepackage[bf,sf]{titlesec}
\usepackage{hyperref}
\usepackage{xurl}
\usepackage[table]{xcolor}
\usepackage{longtable}

\hypersetup{
    colorlinks=true,
    linkcolor=black,
    urlcolor=black
}
\urlstyle{rm}

% --- Dokumenttitel ---
\title{\vspace{-1em}\textbf{Konkrete Umsetzung von PAKETH am D-CHAB}}
\author{Connor Pütz (Präsident) \quad | \quad Paul Gärtner (HoPo-C) \quad | \quad Simon Gläser (HoPo-N)}
\date{\today}

\begin{document}
\maketitle

\tableofcontents
\newpage

% ================================================================
\section{Einleitung}

Das Projekt \textbf{PAKETH} (Prüfungen und Akademischer Kalender an der ETH Zürich) ist eine umfassende Lehr- und Studienreform, mit der die ETH Zürich ihre Studienstrukturen, Prüfungszyklen und Lehrkonzepte modernisieren möchte.  
Ziel des Projekts ist es, den akademischen Kalender zu vereinheitlichen, die Prüfungsphasen zu entzerren, Lehr- und Lernphasen klarer zu strukturieren und die Studierbarkeit der ETH-Programme zu verbessern. Dabei sollen sowohl Studierende als auch Lehrende von einer besseren Planbarkeit und Flexibilität profitieren.  

Im Zentrum von PAKETH stehen drei große Veränderungen:
\begin{itemize}[leftmargin=2em]
    \item eine \textbf{Neugestaltung des akademischen Kalenders}, um Prüfungs- und Lernphasen zu harmonisieren,
    \item die \textbf{Einführung modulbasierter Leistungsnachweise} anstelle großer Endprüfungen,
    \item und die \textbf{Optimierung der Lehr- und Lernbelastung} über das gesamte Semester hinweg.
\end{itemize}

Diese Reform bringt für das \textbf{Departement Chemie und Angewandte Biowissenschaften (D-CHAB)} jedoch besondere Herausforderungen mit sich.  
Das D-CHAB zeichnet sich durch eine intensive, praxisorientierte Ausbildung aus, die von zahlreichen Laborpraktika begleitet wird.  
Während andere Departemente vergleichsweise mehr reine Vorlesungszeit haben, ist der Anteil an verpflichtenden Praktika am D-CHAB deutlich höher.  

Die Umstellung des akademischen Kalenders im Rahmen von PAKETH bedeutet, dass die effektive Lernzeit zwischen den Unterrichtsphasen spürbar kürzer wird.  
Da die Praktika weiterhin einen großen Teil der wöchentlichen Arbeitszeit beanspruchen, entsteht für die Studierenden ein \textbf{sehr volles Semester}, in dem weniger Zeit zum selbstständigen Lernen, Wiederholen und Vertiefen bleibt.  

Diese erhöhte Belastung hat auch soziale und strukturelle Konsequenzen:  
Viele Studierende engagieren sich ehrenamtlich in studentischen Organisationen, Fachvereinen oder Kommissionen oder übernehmen als Teaching Assistants (TAs) wertvolle Aufgaben in der Lehre.  
Wenn durch die neue Struktur weniger zeitliche Freiräume bestehen, besteht die \textbf{Gefahr eines Rückgangs dieses Engagements}, was langfristig die studentische Mitgestaltung und die Qualität der Lehre beeinträchtigen könnte.  

Daher ist es entscheidend, dass PAKETH am D-CHAB nicht einfach nur organisatorisch umgesetzt, sondern inhaltlich \textbf{durchdacht und fachgerecht adaptiert} wird.  
Eine erfolgreiche Umsetzung muss darauf abzielen, die \textbf{Arbeitslast zu reduzieren}, ohne die \textbf{Qualität der Ausbildung zu gefährden}.  
Dies erfordert insbesondere:
\begin{itemize}[leftmargin=2em]
    \item eine bessere Abstimmung zwischen Vorlesungen, Übungen und Praktika,
    \item die inhaltliche Straffung von Lehrveranstaltungen ohne Substanzverlust,
    \item und eine klare Priorisierung der Lernziele in allen Lehrmodulen.
\end{itemize}

Ziel dieses Dokuments ist es, Vorschläge und konkrete Maßnahmen für die Umsetzung von PAKETH am D-CHAB darzulegen.  
Dabei soll der Fokus auf einer qualitativ hochwertigen, aber realistisch gestalteten Ausbildung liegen, die Studierende fordert, aber nicht überfordert, und die Raum für persönliches und akademisches Engagement lässt.

% ================================================================
\section{Änderungen an den Curricula der einzelnen Studiengänge}

\subsection{Chemie BSc}

\renewcommand{\arraystretch}{1.2}

\begin{longtable}{|p{0.9\textwidth}|p{0.1\textwidth}|}
\hline
\rowcolor{gray!15}
\textbf{PAKETH (Vorschlag)} & \textbf{KP} \\
\hline
\endfirsthead

\hline
\rowcolor{gray!15}
\textbf{PAKETH (Vorschlag)} & \textbf{KP} \\
\hline
\endhead

% ============================ Basisjahr ============================
\multicolumn{2}{|l|}{\textbf{a. Module des Basisjahrs (Notengewichte) – 44 KP}} \\ \hline

\textbf{Basisprüfungsgruppe A (Pflichtmodule mit Kompensation – 22 KP)} & \\ \hline
Allgemeine Chemie I Teil AC & 3 \\ \hline
Allgemeine Chemie I Teil OC & 3 \\ \hline
Allgemeine Chemie I Teil PC & 3 \\ \hline
Physik I & 3 \\ \hline
GL Mathematik I Teil Analysis A & 3 \\ \hline
Informatik I & 2 \\ \hline

\textbf{Basisprüfungsgruppe B (Pflichtmodule mit Kompensation – 22 KP)} & \\ \hline
Allgemeine Chemie II Teil AC & 3 \\ \hline
Allgemeine Chemie II Teil OC & 3 \\ \hline
Physikalische Chemie I & 3 \\ \hline
Physik II & 3 \\ \hline
GL Mathematik II Teil Analysis B & 3 \\ \hline
GL Mathematik II LA und Statistik & 2 \\ \hline

% =================== Module höheres Bachelorstudium ===================
\multicolumn{2}{|l|}{\textbf{b. Module höheres Bachelorstudium – 96 KP}} \\ \hline

\textbf{Kernmodulgruppe A (Pflichtmodule mit Kompensation – 17 KP)} & \\ \hline
Anorganische Chemie I & 3 \\ \hline
Organische Chemie I & 4 \\ \hline
Physikalische Chemie II & 3 \\ \hline
Analytische Chemie I & 4 \\ \hline
Mathematik III & 2 \\ \hline

\textbf{Kernmodulgruppe B (Pflichtmodule mit Kompensation – 20 KP)} & \\ \hline
Anorganische Chemie II & 3 \\ \hline
Organische Chemie II & 4 \\ \hline
Physikalische Chemie III & 3 \\ \hline
Analytische Chemie II & 4 \\ \hline
Biologie \textit{(neue Biologie ev. nur noch 4 KP)} & 3 \\ \hline
Chemieingenieurwissenschaften & 3 \\ \hline

\textbf{Kernmodulgruppe C (Pflichtmodule mit Kompensation – 12 KP)} & \\ \hline
Anorganische Chemie III & 3 \\ \hline
Organische Chemie III & 3 \\ \hline
Physikalische Chemie IV & 3 \\ \hline

\textbf{Kernmodulgruppe D (Pflichtmodule mit Kompensation – 16 KP)} & \\ \hline
Anorganische Chemie IV & 3 \\ \hline
Organische Chemie IV & 3 \\ \hline
Physikalische Chemie V & 3 \\ \hline
Sicherheit & 2 \\ \hline

\textbf{Vertiefungsmodule (Wahlpflichtmodule – 25 KP)} & \\ \hline
\textit{Gemäß Wahl der Studierenden} & 25 \\ \hline

\textbf{Wissenschaft im Kontext (WIK) – Wahlpflichtmodule – 6 KP} & \\ \hline
\textit{Gemäß Vorgabe des D-CHAB} & 6 \\ \hline

\textbf{Praxismodule – Pflichtmodule – 50 KP} & \\ \hline
\textit{Gemäß definierter Praktika und Abschlussarbeiten} & 50 \\ \hline
\end{longtable}

% ================================================================
\section{Conclusion}

Die dargestellte Struktur zeigt die neue modulare Organisation des Chemie-Bachelorstudiums unter PAKETH.  
Das Modell fasst verwandte Lehrveranstaltungen zu größeren Modulen zusammen und stärkt die Kohärenz zwischen Theorie und Praxis.  
Für die erfolgreiche Umsetzung sind jedoch gezielte Anpassungen bei Praktika, Prüfungszeitpunkten und Leistungsnachweisen erforderlich, um die Balance zwischen Workload und Qualität zu gewährleisten.

\vfill
\noindent\textbf{Kontakt:}\\
\href{mailto:puetzc@vcs.ethz.ch}{puetzc@vcs.ethz.ch} \quad
\href{mailto:pgaertner@vcs.ethz.ch}{pgaertner@vcs.ethz.ch} \quad
\href{mailto:glaesers@vcs.ethz.ch}{glaesers@vcs.ethz.ch}

\end{document}
