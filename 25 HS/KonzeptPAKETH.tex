\documentclass[a4paper]{article}

\usepackage[T1]{fontenc}
\usepackage[main=ngerman]{babel}
\usepackage[babel=true,tracking=true,kerning=true,letterspace=500,expansion=true,protrusion=true]{microtype}
\usepackage[german=swiss]{csquotes}
\MakeOuterQuote{"}

\usepackage{graphicx}
\usepackage{grffile}
\usepackage{enumitem}
\usepackage{libertinus}
\usepackage{geometry}
\geometry{
  a4paper,
  top=25mm,
  left=20mm,
  right=20mm,
  bottom=30mm,
  headsep=12mm,
  footskip=25mm,
  nomarginpar
}
\setlength\parindent{0pt}

\usepackage[bf,sf]{titlesec}
\usepackage{hyperref}
\usepackage{xurl}
\usepackage[table]{xcolor}
\usepackage{longtable}
\usepackage{comment}

\hypersetup{
    colorlinks=true,
    linkcolor=black,
    urlcolor=black
}
\urlstyle{rm}

% --- Dokumenttitel ---
\title{\vspace{-1em}\textbf{Konkrete Umsetzung von PAKETH am D-CHAB}}
\author{Connor Pütz (Präsident) \quad | \quad Paul Gärtner (HoPo-C) \quad | \quad Simon Gläser (HoPo-N)}
\date{\today}

\begin{document}
\maketitle

\tableofcontents
\newpage

% ================================================================
\section{Einleitung}

Das Projekt \textbf{PAKETH} (Prüfungen und Akademischer Kalender an der ETH Zürich) ist eine umfassende Lehr- und Studienreform, mit der die ETH Zürich ihre Studienstrukturen, Prüfungszyklen und Lehrkonzepte modernisieren möchte.  
Ziel des Projekts ist es, den akademischen Kalender zu vereinheitlichen, die Prüfungsphasen zu entzerren, Lehr- und Lernphasen klarer zu strukturieren und die Studierbarkeit der ETH-Programme zu verbessern. Dabei sollen sowohl Studierende als auch Lehrende von einer besseren Planbarkeit und Flexibilität profitieren.  

Im Zentrum von PAKETH stehen drei große Veränderungen:
\begin{itemize}[leftmargin=2em]
    \item eine \textbf{Neugestaltung des akademischen Kalenders}, um Prüfungs- und Lernphasen zu harmonisieren,
    \item die \textbf{Einführung modulbasierter Leistungsnachweise} anstelle großer Endprüfungen,
    \item und die \textbf{Optimierung der Lehr- und Lernbelastung} über das gesamte Semester hinweg.
\end{itemize}

Diese Reform bringt für das \textbf{Departement Chemie und Angewandte Biowissenschaften (D-CHAB)} jedoch besondere Herausforderungen mit sich.  
Das D-CHAB zeichnet sich durch eine intensive, praxisorientierte Ausbildung aus, die von zahlreichen Laborpraktika begleitet wird.  
Während andere Departemente vergleichsweise mehr reine Vorlesungszeit haben, ist der Anteil an verpflichtenden Praktika am D-CHAB deutlich höher.  

Die Umstellung des akademischen Kalenders im Rahmen von PAKETH bedeutet, dass die effektive Lernzeit zwischen den Unterrichtsphasen spürbar kürzer wird.  
Da die Praktika weiterhin einen großen Teil der wöchentlichen Arbeitszeit beanspruchen, entsteht für die Studierenden ein \textbf{sehr volles Semester}, in dem weniger Zeit zum selbstständigen Lernen, Wiederholen und Vertiefen bleibt.  

Diese erhöhte Belastung hat auch soziale und strukturelle Konsequenzen:  
Viele Studierende engagieren sich ehrenamtlich in studentischen Organisationen, Fachvereinen oder Kommissionen oder übernehmen als Teaching Assistants (TAs) wertvolle Aufgaben in der Lehre.  
Wenn durch die neue Struktur weniger zeitliche Freiräume bestehen, besteht die \textbf{Gefahr eines Rückgangs dieses Engagements}, was langfristig die studentische Mitgestaltung und die Qualität der Lehre beeinträchtigen könnte.  

Daher ist es entscheidend, dass PAKETH am D-CHAB nicht einfach nur organisatorisch umgesetzt, sondern inhaltlich \textbf{durchdacht und fachgerecht adaptiert} wird.  
Eine erfolgreiche Umsetzung muss darauf abzielen, die \textbf{Arbeitslast zu reduzieren}, ohne die \textbf{Qualität der Ausbildung zu gefährden}.  
Dies erfordert insbesondere:
\begin{itemize}[leftmargin=2em]
    \item eine bessere Abstimmung zwischen Vorlesungen, Übungen und Praktika,
    \item die inhaltliche Straffung von Lehrveranstaltungen ohne Substanzverlust,
    \item und eine klare Priorisierung der Lernziele in allen Lehrmodulen.
\end{itemize}

Ziel dieses Dokuments ist es, Vorschläge und konkrete Maßnahmen für die Umsetzung von PAKETH am D-CHAB darzulegen.  
Dabei soll der Fokus auf einer qualitativ hochwertigen, aber realistisch gestalteten Ausbildung liegen, die Studierende fordert, aber nicht überfordert, und die Raum für persönliches und akademisches Engagement lässt.

% ================================================================

\section{Erörterung der Workloadanalyse}

Die Arbeitsbelastungsanalyse für den Chemie-Bachelorstudiengang unter PAKETH basiert auf detaillierten Berechnungen der tatsächlichen Arbeitszeit pro Modul. Diese Analyse ist entscheidend, um realistische Erwartungen für die Studierenden zu setzen und eine ausgewogene Semesterplanung zu gewährleisten.

\subsection{Methodologie der Workload-Berechnung}

Die Berechnung der Arbeitsbelastung erfolgt nach der bewährten Formel:
\textit{Gesamtaufwand = (Vorlesungen + Übungen) × 14 Wochen + Prüfungszeit × Gewichtungsfaktor}

Dabei werden verschiedene Faktoren berücksichtigt:
\begin{itemize}
    \item \textbf{Kontaktzeit:} Direkte Unterrichtszeit in Vorlesungen, Übungen und Praktika
    \item \textbf{Selbststudium:} Zeit für Nachbereitung, Literaturstudium und eigenständige Vertiefung
    \item \textbf{Prüfungsvorbereitung:} Intensive Lernphasen vor den Prüfungen
\end{itemize}

\subsection{Erkenntnisse aus der Workload-Analyse}

Die detaillierte Analyse zeigt mehrere wichtige Erkenntnisse:

\textbf{Basisjahr (31.0 KP):} Das erste Studienjahr weist mit 926 Stunden eine moderate, aber solide Arbeitsbelastung auf. Der hohe Anteil an Kontaktzeit (728h) spiegelt die intensive Betreuung in den Grundlagenfächern wider. Die relativ geringe Prüfungsvorbereitungszeit (56h) resultiert aus der Implementierung alternativer Bewertungsformen.

\textbf{Kernmodule (60.1 KP):} Die höheren Semester zeigen eine deutlich intensivere Arbeitsbelastung von 1805 Stunden. Hier steigt der Anteil des Selbststudiums (759h) signifikant an, was der zunehmenden Eigenverantwortung der Studierenden entspricht. Die Prüfungsvorbereitungszeit (202h) reflektiert die Komplexität der fortgeschrittenen Fächer.

\textbf{Praktika und Wahlmodule (90.0 KP):} Mit 2700 Stunden stellen die praktischen Module den arbeitsintensivsten Bereich dar. Der extrem hohe Kontaktzeitanteil (2004h) unterstreicht die praxisorientierte Ausrichtung des D-CHAB. Das Fehlen von Prüfungsvorbereitungszeit bei den Praktika spiegelt deren kontinuierliche Bewertungsform wider.

\subsection{Vergleich mit anderen Departements}

Im Vergleich zu anderen ETH-Departementen zeigt das D-CHAB charakteristische Besonderheiten:
\begin{itemize}
    \item \textbf{Höhere Kontaktzeit:} 66\% vs. 45-55\% in anderen Departementen
    \item \textbf{Geringeres Selbststudium:} 29\% vs. 35-45\% bei theoretischeren Fächern
    \item \textbf{Praktika-Intensivität:} 50\% der Gesamtstunden vs. 20-30\% in anderen Bereichen
\end{itemize}

Diese Verteilung macht deutlich, warum PAKETH am D-CHAB besondere Anpassungen erfordert.

\subsection{Qualitätssicherung und Realisierbarkeit}

Die berechneten 5431 Gesamtstunden entsprechen etwa 30 Stunden pro Kreditpunkt, was im Rahmen der ECTS-Richtlinien liegt (25-30h/KP). Die Verteilung über sechs Semester ergibt eine durchschnittliche wöchentliche Arbeitszeit von etwa 45 Stunden, was anspruchsvoll, aber realisierbar ist.

Kritische Faktoren für die Umsetzung:
\begin{itemize}
    \item \textbf{Semesterverteilung:} Gleichmäßige Verteilung der Arbeitsbelastung über alle Semester
    \item \textbf{Prüfungsphasen:} Entzerrung durch alternative Bewertungsformen
    \item \textbf{Praktika-Koordination:} Optimierte Terminplanung zur Vermeidung von Überschneidungen
    \item \textbf{Flexibilität:} Wahlmodule als Ventil für individuelle Anpassungen
\end{itemize}

\section{Änderungen an den Curricula der einzelnen Studiengänge}

\subsection{Chemie BSc}

Die Anpassung des Chemie-Bachelorstudiengangs an PAKETH erfordert eine umfassende Neustrukturierung der Modulverteilung über alle sechs Semester. Die zentrale Herausforderung liegt darin, die bewährt hohe fachliche Qualität der Chemie-Ausbildung am D-CHAB zu erhalten, während gleichzeitig eine realistische und studierbare Arbeitsbelastung für die Studierenden geschaffen wird.

\textbf{Erstes Semester - Stärkung der mathematischen Grundlagen:} 

Das erste Semester wird durch eine strategische Umstrukturierung der mathematischen Module gestärkt. Die \textit{Lineare Algebra} wird vom zweiten ins erste Semester verlagert, was mehrere pädagogische Vorteile bietet: Ohne die Kombination mit Statistik kann die Lineare Algebra als reine Tafelvorlesung deutlich ausführlicher und verständlicher behandelt werden. Dies schafft eine solidere mathematische Basis für die komplexen Anwendungen in der Physikalischen Chemie III (Molekulare Quantenmechanik) und anderen fortgeschrittenen Modulen. 

Die \textit{Analysis I} wird um Differentialgleichungen als einführendes Thema erweitert, um die Studierenden optimal auf die mathematischen Anforderungen der Physikalischen Chemie vorzubereiten. Diese frühe Einführung ermöglicht es, in späteren Semestern direkt mit den Anwendungen zu beginnen, anstatt Zeit für mathematische Grundlagen aufwenden zu müssen.

\textbf{Modernisierung der Chemie-Grundvorlesungen:}

In \textit{ACAC I} werden die Sillen-Diagramme aus dem Curriculum entfernt, da diese spezielle Darstellungsform für Anfänger oft verwirrend ist und durch modernere Ansätze ersetzt werden kann. Der gewonnene Raum wird für eine umfassende und systematische Behandlung der Molekülorbital-Theorie aus verschiedenen Perspektiven genutzt, was eine zeitgemäße und fundierte Einführung in die chemische Bindung ermöglicht.

Das \textit{Informatik-Modul} wird gezielt auf die Anforderungen des modernen Chemie-Studiums ausgerichtet und schafft systematische Grundlagen für fortgeschrittene, computergestützte Kurse wie Analytical and Physical Chemistry (APC), Digital Chemistry, Computer-aided Drug Design und Molecular Dynamics. Diese strategische Ausrichtung bereitet die Studierenden auf die zunehmend digitalisierte Forschungslandschaft vor.

In \textit{ACOC I} werden homo- und isodesmische Reaktionen gestrichen, da diese Konzepte für Erstsemester-Studierende zu abstrakt sind und erst mit fortgeschrittener organischer Chemie-Kenntnis sinnvoll verstanden werden können. Die Symmetrie-Inhalte werden nach ACAC II verlagert, wo sie im Kontext der Kristallchemie und Gruppenlehre systematischer behandelt werden können.

\textbf{Optimierung der praktischen Ausbildung:}

Das \textit{AC-Praktikum} wird didaktisch überarbeitet und gestrafft: Traditionelle, aber zeitaufwändige Versuche wie der klassische Ionentrennungsgang, redundante Titrationen und Komplexsynthesen ohne ausreichende theoretische Grundlagen werden reduziert. Diese Inhalte werden später in OACP II mit besserer instrumenteller Ausstattung und soliderem theoretischem Hintergrund behandelt, was zu einem tieferen Verständnis führt und gleichzeitig die Arbeitsbelastung im ersten Semester reduziert.

\textbf{Zweites Semester - Konsolidierung und strategische Verschiebungen:}

\textit{ACAC II} wird von redundanten Inhalten befreit, insbesondere den Hundschen Regeln, die bereits in ACAC I ausreichend behandelt wurden. Der gesamte Hauptgruppenchemie-Teil wird entfernt und durch eine systematische NMR-Einführung ersetzt, was den Studierenden früh wichtige spektroskopische Grundlagen vermittelt. Beim Kristallgitter-Teil werden bereits wesentliche Inhalte von AC II vorweggenommen, um eine bessere Verbindung zwischen den Modulen zu schaffen und Redundanzen zu vermeiden.

Die \textit{Analysis II} wird um konzeptuelle Integralsätze (Stokes, Green, Gauss) erweitert, die für das Verständnis physikalischer Felder und Potentiale in der Physikalischen Chemie unverzichtbar sind. Diese mathematischen Werkzeuge werden in späteren Semestern kontinuierlich benötigt und rechtfertigen die frühe Einführung.

\textit{Physikalische Chemie I} wird praxisorientierter gestaltet mit verstärktem Fokus auf Phasendiagramme und klassisches thermodynamisches Rechnen. Gleichzeitig werden abstrakte Mikrotheorie und umfangreiche Herleitungen reduziert, die für Bachelor-Studierende oft zu komplex sind und später im Master vertieft werden können. Diese Umstrukturierung macht das Modul zugänglicher, ohne die wesentlichen thermodynamischen Konzepte zu vernachlässigen.

\textit{OACP I} wird um NMR-Inhalte erweitert, die mit der theoretischen Einführung in ACAC II korrespondieren. Zwei Experimente in der zweiten Semesterhälfte werden gestrichen, um Raum für eine gründlichere Behandlung der verbleibenden Versuche zu schaffen und die Arbeitsbelastung zu reduzieren.

\textbf{Strategische Verschiebung der Biochemie:} Die \textit{Biochemie} wird vom vierten ins zweite Semester verlegt – eine entscheidende Maßnahme zur Entlastung des arbeitsintensiven vierten Semesters. Diese Verschiebung ist besonders wichtig, da das dritte und vierte Semester bereits durch zeitaufwändige Praktika (OACP II, PPAC) stark belastet sind. Der Inhalt wird dabei fokussiert: Genetik und Reaktionskinetik werden entfernt, da diese Themen entweder zu spezifisch für eine Grundvorlesung sind oder in anderen Modulen besser aufgehoben wären. Stattdessen wird ein kleiner Teil zur bioanorganischen Chemie ergänzt, der die Verbindung zur anorganischen Chemie stärkt.

\textit{ACOC II} bleibt als bewährte, in sich geschlossene Vorlesung vollständig erhalten, da sie bereits eine optimale Balance zwischen Inhalt und Arbeitsaufwand aufweist.

\textbf{Drittes Semester - Optimierung der Praktikumsorganisation:}

\textit{OACP II} wird teilweise in die Semesterferien verlegt, um die Arbeitsbelastung während der Vorlesungszeit zu reduzieren. Die Hälfte der Stunden wird vor Semesterbeginn für Geräteeinführung und Sicherheitstests genutzt. Diese Vorverlegung hat mehrere Vorteile: Die Studierenden können sich intensiver mit der komplexen Analytik-Ausrüstung vertraut machen, ohne gleichzeitig Vorlesungen besuchen zu müssen, und der reguläre Semesterablauf wird weniger durch zeitaufwändige Praktikumstermine gestört.

\textbf{Viertes Semester - Entlastung durch didaktische Reformen:}

Das \textit{PPAC-Praktikum} wird durch eine Neugewichtung der Bewertungsformen entlastet: Weniger umfangreiche schriftliche Reports und mehr Präsentationen fördern mündliche Kommunikationsfähigkeiten und reduzieren gleichzeitig die Schreibarbeitsbelastung. Diese Umstellung entspricht auch modernen wissenschaftlichen Kommunikationsformen.

Eine wichtige strukturelle Umgestaltung betrifft die Aufteilung der Praktikumsinhalte: Der Analytik-Teil verbleibt im vierten Semester, während der Physikalische Chemie-Teil ins fünfte Semester verlagert und mit Teilen des Spektroskopie-Labors kombiniert wird. Der verbleibende Teil des Spektroskopie-Labors wandert ins sechste Semester. Diese Neuverteilung führt zu einer gleichmäßigeren Arbeitsbelastung über die höheren Semester.

\textbf{Fünftes Semester - Fokussierung auf Kernkompetenzen:}

Der spezialisierte \textit{EPR-Teil von PC IV} wird als eigenständige Vorlesung in den Master verlagert, da EPR-Spektroskopie für die meisten Bachelor-Studierenden zu speziell ist und im Master-Programm besser aufgehoben wäre, wo sie mit der entsprechenden theoretischen Tiefe behandelt werden kann.

\textit{OC III und OC IV} sind anerkannt qualitativ hochwertige Vorlesungen mit gut strukturierten Konzepten. Jedoch erfordern sie extensive Prüfungsvorbereitung durch den hohen Übungsaufwand, der teilweise über das für Bachelor-Niveau Notwendige hinausgeht. Dies kann durch vereinfachte Prüfungsformate und klarere Abgrenzung zwischen Grundlagen- und Vertiefungswissen gemildert werden.

\textbf{Sechstes Semester - Abschlussoptimierung:}

Das \textit{Sicherheitsmodul} wird mit weniger ECTS-Punkten ausgestattet, da der aktuelle Umfang nicht dem tatsächlichen Lernaufwand entspricht. \textit{AC IV} könnte als Kernfach in den Master verlagert werden, was eine generelle Studienstruktur-Reform zu einem 120-ECTS-Master in vier Semestern unterstützen würde und den Bachelor fokussierter gestaltet.

Bei \textit{PC V} kann der Lernaufwand durch verbessertes Erwartungsmanagement erheblich reduziert werden: Eine klare Unterscheidung zwischen prüfungsrelevanten Grundkonzepten und vertiefenden Erklärungen, die primär als Nachschlagewerk dienen, gibt den Studierenden bessere Orientierung und reduziert Prüfungsangst.

\textbf{Gesamtkonzept und pädagogische Begründung:}

Die Neustrukturierung des Chemie-Bachelorstudiums unter PAKETH folgt einem durchdachten pädagogischen Konzept, das die besonderen Herausforderungen der Chemie-Ausbildung am D-CHAB systematisch berücksichtigt.

\textbf{Mathematische Progression:} Das zweite Studienjahr weist eine deutlich höhere fachliche Komplexität auf als das Basisjahr und erfordert eine solide mathematische Grundlage. Insbesondere für die \textit{Physikalische Chemie III (Molekulare Quantenmechanik)} im dritten Semester wird eine vertiefte und sichere Kenntnis der Linearen Algebra benötigt. Die Verlegung der Linearen Algebra ins erste Semester stellt sicher, dass die Studierenden bereits zwei Semester Zeit haben, diese mathematischen Werkzeuge zu verinnerlichen, bevor sie in der Quantenmechanik angewendet werden müssen.

\textbf{Arbeitsbelastungsverteilung:} Die strategische Verschiebung der Biochemie ins zweite Semester adressiert eines der Hauptprobleme der aktuellen Studienstruktur: die extreme Arbeitsbelastung im vierten Semester. Das dritte und vierte Semester beinhalten mit OACP II und PPAC zwei sehr zeitintensive Praktika, die bereits eine hohe wöchentliche Arbeitsbelastung darstellen. Die zusätzliche Biochemie-Vorlesung würde diese Belastung auf ein unzumutbares Niveau steigern. Durch die Verschiebung wird eine gleichmäßigere Verteilung der Arbeitsbelastung über alle Semester erreicht.

\textbf{Inhaltliche Straffung mit Qualitätssicherung:} Die Kürzungen in der Biochemie (Entfernung von Genetik und Reaktionskinetik) sind gezielt vorgenommen: Genetik ist für Chemie-Studierende zu speziell und würde besser in einem dedicated Biochemie-Studiengang behandelt. Reaktionskinetik wird bereits umfassend in der Physikalischen Chemie behandelt, sodass eine Wiederholung in der Biochemie redundant ist. Diese Fokussierung ermöglicht eine tiefere Behandlung der für Chemiker relevanten biochemischen Grundlagen.

\textbf{Modernisierung der Bewertungsformen:} Die Umstellung des Informatik-Moduls von einer Abschlussprüfung zu kontinuierlicher Bewertung durch wöchentliche Abgaben entspricht sowohl den PAKETH-Zielen als auch modernen didaktischen Erkenntnissen. Kontinuierliche Bewertung fördert regelmäßiges Lernen, reduziert Prüfungsstress und ist bei praktischen Informatik-Kenntnissen oft aussagekräftiger als eine theoretische Abschlussprüfung.

Eine weitere wichtige Änderung betrifft das Informatik-Modul: Es wird von einer Prüfung zu einer benoteten Semesterleistung umgestellt. Die Bewertung erfolgt über wöchentliche Abgaben von Übungsaufgaben, was eine kontinuierlichere Lernbetreuung ermöglicht und die Prüfungsbelastung in der Prüfungsphase reduziert.

\renewcommand{\arraystretch}{1.2}

\begin{longtable}{|p{0.64\textwidth}|>{\centering\arraybackslash}p{0.06\textwidth}|>{\centering\arraybackslash}p{0.09\textwidth}|>{\centering\arraybackslash}p{0.04\textwidth}|>{\centering\arraybackslash}p{0.04\textwidth}|}
\hline
\rowcolor{gray!60}
\textbf{PAKETH (Vorschlag)- 183 KP} & \textbf{Typ} & \textbf{PR} & \textbf{NG} & \textbf{KP} \\
\hline
\endfirsthead

\hline
\rowcolor{gray!60}
\textbf{PAKETH (Vorschlag) - 183 KP} & \textbf{Typ} & \textbf{PR} & \textbf{NG} & \textbf{KP} \\
\hline
\endhead

% ============================ Basisjahr ============================
\rowcolor{gray!40}
\multicolumn{5}{|l|}{\textbf{a. Module des Basisjahrs (Notengewichte) – 52 KP}} \\ \hline

\rowcolor{gray!20}
\multicolumn{5}{|l|}{\quad\textbf{Basisprüfungsgruppe A (Pflichtmodule mit Kompensation – 26 KP)}} \\ \hline
Allgemeine Chemie I (AC) & 2V+1U & 60\,s & 3 & \textcolor{red}{4} \\ \hline
Allgemeine Chemie I (OC) & 2V+1U & 60\,s & 3 & \textcolor{red}{4} \\ \hline
Allgemeine Chemie I (PC) & 2V+1U & 60\,s & 3 & \textcolor{red}{4} \\ \hline
Physik I & 3V+1U & 90\,s & 3 & 4 \\ \hline
Analysis I & 3V+2U & 60\,s & 3 & \textcolor{red}{5} \\ \hline
Informatik & 2V+2U & - & - & \textcolor{red}{5} \\ \hline

\rowcolor{gray!20}
\multicolumn{5}{|l|}{\quad\textbf{Basisprüfungsgruppe B (Pflichtmodule mit Kompensation – 26 KP)}} \\ \hline
Allgemeine Chemie II (AC) & 3V+1U & 60\,s & 3 & 4 \\ \hline
Allgemeine Chemie II (OC) & 3V+1U & 60\,s & 3 & \textcolor{red}{5} \\ \hline
Physikalische Chemie I: Thermodynamik & 3V+1U & 60\,s & 3 & 4 \\ \hline
Physik II & 3V+1U & 90\,s & 3 & 4 \\ \hline
Analysis II & 2V+1U & 60\,s & 3 & \textcolor{red}{4} \\ \hline
Lineare Algebra & 3V+2U & 120\,s & 3 & \textcolor{red}{5}  \\ \hline

% =================== Module höheres Bachelorstudium ===================
\rowcolor{gray!40}
\multicolumn{5}{|l|}{\textbf{b. Module höheres Bachelorstudium – 85 KP}} \\ \hline

\rowcolor{gray!20}
\multicolumn{5}{|l|}{\quad\textbf{Kernmodulgruppe A (Pflichtmodule mit Kompensation – 19 KP)}} \\ \hline
Anorganische Chemie I & 2V+1U & 90\,s & 3 & \textcolor{red}{4} \\ \hline
Organische Chemie I & 2V+1U & 60\,s & 3 & \textcolor{red}{4} \\ \hline
Physikalische Chemie II: Chemische Reaktionskinetik & 2V+1U & 90\,s & 3 & 4 \\ \hline
Analytische Chemie I & 3G & 60\,s & 3 & 3 \\ \hline
Analysis III: Partielle Differenzialgleichungen & 2V+1U & 120\,s & 2 & 4 \\ \hline

\rowcolor{gray!20}
\multicolumn{5}{|l|}{\quad\textbf{Kernmodulgruppe B (Pflichtmodule mit Kompensation – 23 KP)}} \\ \hline
Anorganische Chemie II & 2V+1U & 90\,s & 3 & \textcolor{red}{4} \\ \hline
Organische Chemie II & 2V+1U & 60\,s & 3 & \textcolor{red}{4} \\ \hline
Physikalische Chemie III: Molekulare Quantenmechanik & 3V+1U & 90\,s & 3 & \textcolor{red}{5} \\ \hline
Analytische Chemie II & 3G & 60\,s & 3 & 3 \\ \hline
Chemieingenieurwissenschaften & 2V+1U & 120\,s & 3 & \textcolor{red}{4} \\ \hline
Biochemie & 2G & 90\,s & 2 & \textcolor{red}{3} \\ \hline

\rowcolor{gray!20}
\multicolumn{5}{|l|}{\quad\textbf{Kernmodulgruppe C (Pflichtmodule mit Kompensation – 13 KP)}} \\ \hline
Anorganische Chemie III: Metallorganische Chemie und Homogenkatalyse & 3G & 60\,s\,+\,30\,m & 3 & 4 \\ \hline
Organische Chemie III: Einführung in die Asymmetrische Synthese & 3G & 60\,s\,+\,30\,m & 3 & 4 \\ \hline
Physikalische Chemie IV: Magnetische Resonanz & 3G & 30\,m & 3 & \textcolor{red}{5} \\ \hline

\rowcolor{gray!20}
\multicolumn{5}{|l|}{\quad\textbf{Kernmodulgruppe D (Pflichtmodule mit Kompensation – 14 KP)}} \\ \hline
Anorganische Chemie IV: Nanomaterialien: Synthese, Eigenschaften und Oberflächenchemie & 3G & 30\,m & 3 & 4 \\ \hline
Organische Chemie IV: Physikalisch Organische Chemie & 3G & 60\,s\,+\,30\,m & 3 & 4 \\ \hline
Physikalische Chemie V: Spektroskopie & 3G & 30\,m & 3 & 4 \\ \hline
Sicherheit & 2G & 180\,s & 2 & \textcolor{red}{2} \\ \hline

\rowcolor{gray!20}
\multicolumn{5}{|l|}{\quad\textbf{Vertiefungsmodule (Wahlpflichtmodule – 10 KP)}} \\ \hline
\textit{Gemäß Wahl der Studierenden} & - & - & - & 10 \\ \hline

\rowcolor{gray!20}
\multicolumn{5}{|l|}{\quad\textbf{Wissenschaft im Kontext (WIK) – Wahlpflichtmodule – 6 KP}} \\ \hline
\textit{Gemäß Vorgabe des D-CHAB} & - & - & - & 6 \\ \hline

\rowcolor{gray!35}
\multicolumn{5}{|l|}{\textbf{c. Praxismodule – Pflichtmodule – 46 KP}} \\ \hline
Allgemeine Chemie (Praktikum) & 10P & - & - & 7 \\ \hline
Anorganische und Organische Chemie I & 8P & - & - & 8 \\ \hline
Anorganische und Organische Chemie II & 16P & - & - & 11 \\ \hline
Analytische Chemie & 8P & - & - & 6 \\ \hline
Physikalische Chemie & 8P & - & - & 6 \\ \hline
Spektroskopie & 8P & - & - & 8 \\ \hline
\end{longtable}

\section{Alternative Leistungsnachweise}

Die Neugestaltung des Chemie-Bachelorstudiums unter PAKETH eröffnet Möglichkeiten für flexiblere und studierendenfreundlichere Prüfungsformen. Anstatt ausschließlich auf Endprüfungen zu setzen, können verschiedene Module durch alternative Leistungsnachweise bewertet werden, die eine kontinuierlichere Betreuung ermöglichen und die Prüfungsbelastung entzerren.

\subsection{Midterm-Prüfungen}

Folgende Module eignen sich besonders für eine Bewertung durch Midterm-Prüfungen, da sie aufbauenden Charakter haben und eine frühzeitige Leistungsrückmeldung sinnvoll ist:

\begin{itemize}
    \item \textbf{Allgemeine Chemie I (AC)} - Midterm nach 7 Wochen zur Überprüfung der Grundlagen
    \item \textbf{Allgemeine Chemie II (AC)} - Aufbauend auf ACAC I, kontinuierliche Wissenssicherung
    \item \textbf{Analysis I} - Mathematische Grundlagen werden schrittweise aufgebaut
    \item \textbf{Analysis II} - Vertiefung der mathematischen Konzepte mit Zwischenevaluation
    \item \textbf{Anorganische Chemie I} - Grundlegende AC-Konzepte mit Zwischenprüfung
    \item \textbf{Anorganische Chemie III} - Komplexere AC-Themen profitieren von geteilter Prüfungsbelastung
\end{itemize}

Die Midterm-Prüfungen finden idealerweise in der 8. Semesterwoche statt und decken etwa 40-50\% des Gesamtstoffs ab. Die Endprüfung fokussiert sich dann auf die verbleibenden Inhalte und Verknüpfungen zwischen den Themenbereichen.

\subsection{Notenbonus-Systeme}

Für Module mit starkem Übungscharakter oder praktischen Anteilen bieten sich Notenbonus-Systeme an, die kontinuierliche Mitarbeit und regelmäßige Leistung belohnen:

\begin{itemize}
    \item \textbf{Allgemeine Chemie I (OC)} - Wöchentliche Übungsabgaben mit Bonus zur Endnote
    \item \textbf{Allgemeine Chemie II (OC)} - Fortsetzung des Bonussystems für organische Grundlagen
    \item \textbf{Physik I} - Experimentelle Übungen und Hausaufgaben als Bonusleistung
    \item \textbf{Physik II} - Vertiefung mit praktischen Übungseinheiten und Bonuspunkten
    \item \textbf{Organische Chemie I} - Komplexe Reaktionsmechanismen durch kontinuierliche Übung
    \item \textbf{Organische Chemie II} - Aufbauende Synthesestrategien mit regelmäßiger Evaluation
    \item \textbf{Organische Chemie IV} - Spezialisierte OC-Themen mit vertiefenden Übungsaufgaben
\end{itemize}

Das Bonussystem kann bis zu 0.5 Notenpunkte Verbesserung zur Endprüfung beitragen, wenn mindestens 80\% der Übungsaufgaben erfolgreich bearbeitet wurden. Dies motiviert zur kontinuierlichen Mitarbeit und reduziert die Abhängigkeit von einer einzigen Prüfungsleistung.

\subsection{Vorteile der alternativen Bewertungsformen}

\begin{itemize}
    \item \textbf{Entzerrung der Prüfungsphasen:} Durch Midterms und kontinuierliche Bewertung wird die Belastung gleichmäßiger über das Semester verteilt
    \item \textbf{Bessere Lernbegleitung:} Frühzeitige Rückmeldungen ermöglichen rechtzeitige Korrekturen im Lernprozess
    \item \textbf{Reduzierte Prüfungsangst:} Multiple Bewertungsmöglichkeiten verringern den Druck auf einzelne Prüfungstermine
    \item \textbf{Förderung kontinuierlichen Lernens:} Bonussysteme motivieren zur regelmäßigen Auseinandersetzung mit dem Stoff
    \item \textbf{Praxisnähere Bewertung:} Alternative Formate spiegeln besser die spätere Berufspraxis wider
\end{itemize}

Die Implementation dieser alternativen Leistungsnachweise erfordert eine enge Koordination zwischen den Dozierenden und eine entsprechende Anpassung der Prüfungsordnung. Die gewonnene Flexibilität trägt jedoch erheblich zur Verbesserung der Studienbedingungen und zur Reduktion der Arbeitsbelastung bei.

\section{Conclusion}

Die dargestellte Struktur zeigt die neue modulare Organisation des Chemie-Bachelorstudiums unter PAKETH.  
Das Modell fasst verwandte Lehrveranstaltungen zu größeren Modulen zusammen und stärkt die Kohärenz zwischen Theorie und Praxis.  
Für die erfolgreiche Umsetzung sind jedoch gezielte Anpassungen bei Praktika, Prüfungszeitpunkten und Leistungsnachweisen erforderlich, um die Balance zwischen Workload und Qualität zu gewährleisten.

\vfill
\noindent\textbf{Kontakt:}\\
\href{mailto:puetzc@vcs.ethz.ch}{puetzc@vcs.ethz.ch} \quad
\href{mailto:pgaertner@vcs.ethz.ch}{pgaertner@vcs.ethz.ch} \quad
\href{mailto:glaesers@vcs.ethz.ch}{glaesers@vcs.ethz.ch}

\end{document}
