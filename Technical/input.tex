%Welcome :)

\documentclass{article}

% Basic document formatting
\usepackage[utf8]{inputenc}     % Input encoding
\usepackage[T1]{fontenc}        % Font encoding
\usepackage{lmodern}            % Modern LaTeX fonts
\usepackage{geometry}           % Set page margins
\geometry{a4paper, total={170mm,257mm}, left=20mm, top=20mm} 
\usepackage{float}              % Handling of floating elements
\usepackage{fancyhdr}           % Fancy headers
\usepackage{lastpage}           % use \pageref{LastPage} to make page x of y footers
\setlength{\parindent}{0pt}     % No \noindent

% Figures
\usepackage{graphicx}           % For including images
\usepackage{caption}            % Using the caption package
\usepackage{wrapfig}            % For including Wrap Figures
\usepackage{subcaption}         % For subfigures within a figure environment
\usepackage{pgfplots}           % Drawing plots
\usepackage{pgf-pie}            % For creating pie charts
\captionsetup[figure]{labelfont=bf}
\captionsetup[table]{labelfont=bf}
\usepackage{asymptote}          % Zum Zeichnen verschiedener Plots 
\usepackage{pdfpages}           % Zum Einfügen ganzer PDFs

% Colorboxes
\usepackage[skins]{tcolorbox}   % Color Boxes

% Tables and long tables
\usepackage{tabularx}           % Advanced table features
\usepackage{longtable}          % For tables that span multiple pages
\usepackage{multirow}           % Allows for multirow cells in tables
\usepackage{booktabs}           % For professional-quality tables

% Math packages
\usepackage{amsmath}            % Enhanced mathematical formatting
\usepackage{amssymb}            % Extended symbol collection
\usepackage{amsfonts}           % Mathematical fonts
\usepackage[version=4]{mhchem}  % Chemische Formeln
\usepackage{mathtools}          % Mathematical tools to supplement amsmath
\numberwithin{equation}{section} % Numbers Equations with chapters
\usepackage{siunitx}            % Makes SI-Units

% Code display
\usepackage{listings}           % For displaying code
\usepackage{xcolor}             % For coloring code
\lstdefinestyle{mystyle}{
    backgroundcolor=\color{backcolour},   
    commentstyle=\color{codegreen},
    keywordstyle=\color{ao},
    numberstyle=\tiny\color{codegray},
    basicstyle=\ttfamily\footnotesize,
    breakatwhitespace=false,         
    breaklines=true,                 
    captionpos=b,                    
    keepspaces=true,                 
    numbers=left,                    
    numbersep=5pt,                  
    showspaces=false,                
    showstringspaces=false,
    showtabs=false,                  
    tabsize=2
}
\lstset{style=mystyle}

% Custom Colours
\definecolor{LightCyan}{rgb}{0.88,1,1}
\definecolor{dkgreen}{rgb}{0,0.6,0}
\definecolor{gray}{rgb}{0.5,0.5,0.5}
\definecolor{mauve}{rgb}{0.58,0,0.82}
\definecolor{codegreen}{rgb}{0,0.6,0}
\definecolor{codegray}{rgb}{0.5,0.5,0.5}
\definecolor{ao}{rgb}{0.0, 0.0, 1.0}
\definecolor{backcolour}{rgb}{0.95,0.95,0.92}

% Referencing
\usepackage[style=numeric, backend=biber, sorting=none]{biblatex} % Imports biblatex package
\addbibresource{MAIN.bib} %Import the bibliography file
\DeclareFieldFormat{labelnumberwidth}{\mkbibbrackets{#1}} %ensure that the label numbers in the bibliography are enclosed in brackets.
\usepackage{xurl}

% Hyperlinks in the document
\usepackage{hyperref}           % For adding hyperlinks